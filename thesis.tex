%
% Template Laporan Skripsi/Thesis Universitas Indonesia
%
% @author  Ichlasul Affan, Azhar Kurnia
% @version 2.1.2
%
% Dokumen ini dibuat berdasarkan standar IEEE dalam membuat class untuk
% LaTeX dan konfigurasi LaTeX yang digunakan Fahrurrozi Rahman ketika
% membuat laporan skripsi, yang kemudian diadaptasi oleh Andreas Febrian dan
% Lia Sadita untuk template skripsi tahun 2010.
% Konfigurasi template sebelumnya telah disesuaikan dengan
% aturan penulisan thesis yang dikeluarkan UI pada tahun 2017.
%

%
% Tipe dokumen adalah report dengan satu kolom.
%
\documentclass[12pt, a4paper, onecolumn, twoside, final]{report}
\raggedbottom

% Load konfigurasi LaTeX untuk tipe laporan thesis
\usepackage{_internals/uithesis}
%

\usepackage[]{minted}
\usepackage{tcolorbox}
\usepackage{etoolbox}
\usepackage[T1]{fontenc}
\usepackage{inconsolata}

\usepackage{titlesec}
\usepackage{natbib}
\usepackage{multirow}
\titleformat{\section}
  {\normalfont\fontsize{12}{15}\bfseries}{\thesection}{0.5em}{}
\titleformat{\subsection}
  {\normalfont\fontsize{12}{15}\bfseries}{\thesubsection}{0.5em}{}
\titleformat{\subsubsection}
  {\normalfont\fontsize{12}{15}\bfseries}{\thesubsubsection}{0.5em}{}

\newcommand{\listmintedcodename}{Daftar Kode Program}
\newlistof{mintedcode}{mcode}{\listmintedcodename}

% (Bahasa, file, caption)
\newcommand{\mintedcode}[3]{%
    \refstepcounter{mintedcode}
    \begin{tcolorbox}[boxrule=0.5pt,leftrule=0.5pt,arc=0pt,auto outer arc]
        \setstretch{0.9}
        \inputminted[]{#1}{#2}
    \end{tcolorbox}
    \begin{@empty}
        \setlength\topsep{0pt}
        \setlength\parskip{0pt}
        \begin{center}
            \par\noindent\textbf{Kode \thechapter.\themintedcode. #3}
        \end{center}
    \end{@empty}
    \addcontentsline{mcode}{mintedcode}
    {\protect\numberline{\thechapter.\themintedcode}#3}\par
}

% Load konfigurasi khusus untuk laporan yang sedang dibuat
%-----------------------------------------------------------------------------%
% Informasi Mengenai Dokumen
%-----------------------------------------------------------------------------%
%

% Judul laporan.
\def\judul{PeerToCP: \textit{Local-first Real-time Collaborative Code Editor} Berbasis {WebRTC}}
%
% Tulis kembali judul laporan, kali ini akan diubah menjadi huruf kapital
\Var{\Judul}{PeerToCP: \textit{Local-first Real-time Collaborative Code Editor} Berbasis {WebRTC}}
%
% Tulis kembali judul laporan namun dengan bahasa Ingris
\def\judulInggris{PeerToCP: WebRTC Based Local-first Real-time Collaborative Code Editor}

%
% Tipe laporan, dapat berisi: Laporan Kerja Praktik, Skripsi, Tugas Akhir, Thesis, atau Disertasi
\def\type{Skripsi}
%
% Jenjang studi, dapat berisi: Diploma, Sarjana, Magister, atau Doktor
\def\jenjang{Sarjana}
%
% Tulis kembali tipe laporan, kali ini akan diubah menjadi huruf kapital
\Var{\Type}{Skripsi}
%
% Tulis nama penulis
\def\penulis{Hocky Yudhiono}
%
% Tulis kembali nama penulis, kali ini akan diubah menjadi huruf kapital
\Var{\Penulis}{Hocky Yudhiono}
%
% Tulis NPM penulis
\def\npm{1906285604}
%
% Tuliskan Fakultas dimana penulis berada
\Var{\Fakultas}{Ilmu Komputer}
\def\fakultas{Ilmu Komputer}
%
% Tuliskan Program Studi yang diambil penulis
\Var{\Program}{Ilmu Komputer}
\def\program{Ilmu Komputer}
% Program Studi dalam bahasa inggris
\def\studyProgram{Computer Science}
%
% Tuliskan tahun publikasi laporan
\Var{\bulanTahun}{Desember 2022}
%
% Tuliskan gelar yang akan diperoleh dengan menyerahkan laporan ini
\def\gelar{Sarjana Ilmu Komputer}
%
% Tuliskan tanggal pengesahan laporan, waktu dimana laporan diserahkan ke
% penguji/sekretariat
\def\tanggalSiapSidang{20 November 2022}
%
% Tuliskan tanggal keputusan sidang dikeluarkan dan penulis dinyatakan
% lulus/tidak lulus
\def\tanggalLulus{1 Desember 2022}
%
% Tuliskan pembimbing
\def\pembimbingSatu{Muhammad Hafizhuddin Hilman S.Kom., M.Kom., Ph.D.}
% S1 s.d. S3: Kosongkan jika tidak ada pembimbing kedua
\def\pembimbingDua{}
% S2 & S3: Kosongkan jika tidak ada pembimbing ketiga
\def\pembimbingTiga{}
%
% Tuliskan penguji
\def\pengujiSatu{Penguji Pertama Anda}
\def\pengujiDua{Penguji Kedua Anda}
% Kosongkan jika tidak ada penguji ketiga (umumnya penguji ketiga hanya ada untuk S2)
\def\pengujiTiga{}
% Kosongkan jika tidak ada penguji keempat, kelima, atau keenam (umumnya penguji > 3 hanya ada untuk S3)
\def\pengujiEmpat{}
\def\pengujiLima{}
\def\pengujiEnam{}

%-----------------------------------------------------------------------------%
% Judul Setiap Bab
%-----------------------------------------------------------------------------%
%
% Berikut ada judul-judul setiap bab.
% Silahkan diubah sesuai dengan kebutuhan.
%
\Var{\kataPengantar}{Kata Pengantar}
\Var{\babSatu}{Pendahuluan}
\Var{\babDua}{Tinjauan Pustaka}
\Var{\babTiga}{Metodologi Penelitian}
\Var{\babEmpat}{Desain dan Implementasi}
\Var{\babLima}{Hasil dan Pembahasan}
\Var{\kesimpulan}{Penutup}

% Daftar pemenggalan suku kata dan istilah dalam LaTeX
%
% Hyphenation untuk Indonesia
%
% @author  Andreas Febrian
% @version 2.1.2
% @edit by Ichlasul Affan, Muhammad Aulia Adil Murtito
%
% Tambahkan cara pemenggalan kata-kata yang salah dipenggal secara otomatis
% oleh LaTeX. Jika kata tersebut dapat dipenggal dengan benar, maka tidak
% perlu ditambahkan dalam berkas ini. Tanda pemenggalan kata menggunakan
% tanda '-'; contoh:
% menarik
%   --> pemenggalan: me-na-rik
%


% Silakan ganti ke bahasa Inggris (\selectlanguage{english}) jika Anda merasa terlalu banyak kata bahasa Inggris yang pemenggalannya tidak benar.
%\selectlanguage{english}


\hyphenation{
    % alphabhet A
    a-na-li-sa a-tur a-tur-an
    a-pli-ka-si a-pli-ka-si-nya
    ak-ti-vi-tas
    % alphabhet B
    bab ba-ngun-an
    be-be-ra-pa
    ber-ge-rak
    ber-ke-lan-jut-an
    ber-o-per-ra-si
    ber-pe-nga-ruh
    % alphabhet C
    ca-ca-han
    chan-nel
    con-nec-tiv-i-ty
    ca-ri Com-po-nent-UML
    % alphabhet D
    di-ban-ding-kan
    di-da-pat-kan di-sim-pan di-pim-pin di-tam-bah-kan di-tem-pat-kan de-ngan da-e-rah di-ba-ngun di-gu-na-kan da-pat di-nya-ta-kan
    di-se-mat-kan di-sim-bol-kan di-pi-lih di-li-hat de-fi-ni-si di-de-fi-ni-si-kan di-mo-del-kan di-mi-li-ki di-re-a-li-sa-si-kan di-su-sun
    di-te-rap-kan
    di-se-le-sai-kan
    di-ap-li-ka-si-kan
    % alphabhet E
    eks-pli-sit e-ner-gi en-gi-neer en-gi-neer-ing eks-klu-sif ele-men
    es-tab-lish-ment en-vi-ron-ment
    eks-pe-ri-men
    % alphabhet F
    fa-si-li-tas
    front-end
    % alphabhet G
    ga-bung-an ge-rak ge-ne-ral ge-ne-ra-li-sa-si
    % alphabhet H
    ha-lang-an
    % alphabhet I
    in-fra-struk-tur i-ni-si-a-si
    % alphabhet J
    % alphabhet K
    ko-la-bo-ra-tif
    ke-hi-lang-an
    ke-ter-hu-bung-an
    ku-ning
    kua-li-tas ka-me-ra ke-mung-kin-an ke-se-pa-ham-an
    % alphabhet L
    ling-kung-an
    lo-gi-cal
    % alphabhet M
    ma-na-je-men me-neng-ah meng-a-da-kan me-mo-ni-tor
    me-mer-lu-kan me-mo-del-kan men-de-fi-ni-si-kan meng-ak-ses me-ne-mu-kan
    meng-a-tas-i me-mo-di-fi-ka-si me-mung-kin-kan me-nge-na-i me-ngi-rim-kan meng-i-zin-kan
    meng-u-bah meng-a-dap-ta-si me-nya-ta-kan me-nyim-pan me-res-trik-si mi-cro-ser-vi-ce mi-cro-ser-vi-ces mo-di-fi-ka-si mo-dul mo-dule
    meng-a-tur meng-a-rah-kan mi-lik meng-gu-na-kan me-ne-ri-ma me-nga-la-mi
    me-di-a-stream-track
    me-di-a-stream
    me-mo-ri
    me-ne-rang-kan
    % alphabhet N
    nya-ta non-eks-klu-sif né-de-lec
    % alphabhet O
    o-pe-ra-si or-ga-ni-sa-si
    o-pe-ra-si-nya
    % alphabhet P
    pa-ling
    pen-ding
    pe-nyim-pa-nan
	pe-nye-rap-an
    peer-to-cp
	pe-ngon-trol
    pe-mo-del-an
    per-tim-ba-ngan
    per-ban-di-ngan
    pe-nga-la-man
    pe-ran  pe-ran-an-nya
    pem-ba-ngun-an pre-si-den pe-me-rin-tah pe-mi-li-han prio-ri-tas peng-am-bil-an
    peng-ga-bung-an pe-nga-was-an pe-ngem-bang-an
    pe-nga-ruh pe-nge-lo-la pa-ra-lel-is-me per-hi-tung-an per-ma-sa-lah-an
    pen-ca-ri-an pen-ce-ta-kan peng-struk-tur-an pen-ting pen-ting-nya pe-ngu-ku-ran
    pre-sen-ta-si pe-nyo-co-kan
    % alphabhet Q
    % alphabhet R
    ru-ang-an
    rep-li-ka
    ran-cang-an re-fe-ren-si re-pre-sen-ta-si
    rtc-da-ta-chan-nel
    ru-a-ngan
    % alphabhet S
    su-a-tu
    sub-bab si-mu-la-si sa-ngat ska-la-bi-li-tas
    stan-dar-di-sa-si sig-nalling
    sa-tu-an
    ser-ver
    se-be-ra-pa
    sig-nal-ing
    % alphabhet T
    te-ngah
    ter-da-pat
    trans-for-ma-si
    % alphabhet U
    u-ti-li-sa-si
    % alphabhet V
    va-li-da-si va-ri-an va-ri-a-si va-ri-a-bi-li-tas ve-ri-fi-ka-si
    % alphabhet W
    web-rtc
    % alphabhet X
    % alphabhet Y
    % alphabhet Z
    % special
}

% Daftar istilah yang mungkin perlu ditandai
\input{config/istilah}

\renewenvironment{newminted}[2]% environment name
{% begin code
 
}%
{% end code
}

% Awal bagian penulisan laporan
\begin{document}
%
% Sampul Laporan
\include{_internals/sampul}
\forceclearchapter

%
% Gunakan penomeran romawi
\pagenumbering{roman}
%
% Menghilangkan penebalan pada huruf-huruf table of content
% dari halaman judul hingga daftar lampiran
\disableboldchapterintoc
%
% load halaman judul dalam
\addChapter{HALAMAN JUDUL}
\include{_internals/judul_dalam}
\forceclearchapter

%
% load halaman orisinalitas

% Menghilangkan penomoran
\pagenumbering{gobble}

\strcompare{Laporan Kerja Praktik}{\type}{}
{
	\include{src/00-frontMatter/pernyataanOrisinalitas}
	\forceclearchapter
}

% Memunculkan penomoran kembali
\pagenumbering{roman}

%
% setelah bagian ini, halaman dihitung sebagai halaman ke 2
\setcounter{page}{2}

%
% Lembar Penegesahan
\strcompare{Laporan Kerja Praktik}{\type}
{
	% Lembar Pengesahan Kerja Praktik dari LaTeX
	\addChapter{LEMBAR PERSETUJUAN DOSEN KERJA PRAKTIK}
	\include{src/00-frontMatter/pengesahanKP}
	\forceclearchapter
}
{
	\addChapter{LEMBAR PENGESAHAN}
	% Gunakan salah satu (comment atau hapus kode yang tidak perlu):
	% Lembar Pengesahan Tugas Akhir dari LaTeX
	\strcompare{Doktor}{\jenjang}
	{\include{src/00-frontMatter/pengesahanSidangS3}}
	{\include{src/00-frontMatter/pengesahanSidang}}
	\forceclearchapter
	% Lembar Pengesahan dari PDF lain (misal: generated oleh SISIDANG [Fasilkom])
	%\putpdf{assets/pdfs/pengesahanSidang.pdf}
}


\strcompare{Laporan Kerja Praktik}{\type}{}
{
	%
	% Kata Pengantar
	\addChapter{KATA PENGANTAR}
	%-----------------------------------------------------------------------------%
\chapter*{\kataPengantar}
%-----------------------------------------------------------------------------%
\pagestyle{first-pages}

Puji syukur kita panjatkan ke hadirat Tuhan Yang Maha Esa, karena rahmat dan anugerah-Nya, penulis dapat menyelesaikan skripsi yang berjudul “\judul” yang menjadi salah satu syarat kelulusan dalam menempuh pendidikan Sarjana Ilmu Komputer di  Fakultas Ilmu Komputer, Universitas Indonesia. Penulis juga ingin berterima kasih kepada pihak-pihak lain, khususnya kepada:

\begin{itemize}
    \setlength\itemsep{-0.5em}
    \item kedua orang tua dan keluarga penulis yang mendukung proses perkuliahan sembari menyelesaikan skripsi ini;
    \item Bapak Muhammad Hafizhuddin Hilman S.Kom., M.Kom., Ph.D. selaku dosen pembimbing tugas akhir yang senantiasa memberikan dukungan mental dan pengetahuan;
    \item Ibu Dr. Putu Wuri Handayani, S.Kom., M.Sc., Ibu Dr. Eng. Laksmita Rahadianti S.Kom., M.Sc., dan Ibu Annisa Monicha Sari, S.Kom., M.Kom. selaku dosen yang membimbing dan memberikan ilmu metodologi penelitian dan penulisan ilmiah;
    \item serta teman-teman penulis: Kenta, Irfancen, Raihan, Adit, Adimas, Sena, Kak Prabowo, Pikatan, dan teman-teman lain yang tidak dapat penulis sebutkan satu per satu namanya, karena setia menemani dan memberikan dukungan mental kepada penulis.
\end{itemize}

Penulis juga menyadari bahwa masih terdapat kesalahan dan kekurangan dalam penulisan karya ilmiah ini. Penulis berharap karya tulis ini dapat memberikan manfaat dan inspirasi untuk pengembangan dan peradaban ilmu pengetahuan teknologi dan informatika dunia, terutama bangsa Indonesia.

\vspace*{0.1cm}
\begin{flushright}
Depok, \tanggalSiapSidang\\[0.1cm]
\vspace*{1.5cm}
\penulis

\end{flushright}

	\forceclearchapter
	%
	% Lembar Persetujuan Publikasi Ilmiah
	\addChapter{LEMBAR PERSETUJUAN PUBLIKASI ILMIAH}
	\include{src/00-frontMatter/persetujuanPublikasi}
	\forceclearchapter
}

%
% Untuk halaman pertama setiap chapter mulai dari abstrak, tetap berikan mark universitas.
%
\pagestyle{first-pages}

%
\addChapter{ABSTRAK}
%
% Halaman Abstrak
%
% @author  Andreas Febrian
% @version 2.1.2
% @edit by Ichlasul Affan
%

\chapter*{Abstrak}
\singlespacing

\vspace*{0.2cm}

% Untuk conditional statement pembimbing dua
\def\blank{}

\noindent \begin{tabular}{l l p{10cm}}
	Nama&: & \penulis \\
	Program Studi&: & \program \\
	Judul&: & \judul \\
	Pembimbing&: & \pembimbingSatu \\
	\ifx\blank\pembimbingDua
    \else
        \ &\ & \pembimbingDua \\
    \fi
    \ifx\blank\pembimbingTiga
    \else
    	\ &\ & \pembimbingTiga \\
    \fi
\end{tabular} \\

\vspace*{0.5cm}

\noindent Aplikasi kolaboratif dalam waktu nyata menjadi bagian dari kehidupan manusia modern saat ini. Teknologi ini digunakan terutama untuk berkomunikasi dan meningkatkan produktivitas, dengan mengurangi hambatan ruang dan waktu. Salah satu sistem aplikasi yang marak digunakan adalah editor dokumen yang dapat digunakan beberapa pengguna secara bersamaan. Dilatar belakangi oleh motivasi tersebut, penelitian ini memaparkan sebuah aplikasi editor kode kolaboratif \textit{local-first} berbasis \textit{peer-to-peer} yang diimplementasi dengan WebRTC dan CRDT. Selain itu, aplikasi ini menyertai \textit{shell} bersama yang dapat dijalankan oleh salah satu pengguna dan digunakan oleh setiap pengguna lain dalam suatu kelompok jaringan. Terdapat beberapa variasi arsitektur jaringan aplikasi yang menyusun kerangka \textit{backend} aplikasi ini untuk dibandingkan. Dari segi algoritma dalam menjaga konsistensi dokumen, dua pendekatan berbeda dievaluasi dalam penelitian ini, yakni algoritma OT (\textit{operational transformation}) dan metode yang memanfaatkan struktur data CRDT (\textit{conflict-free replicated data types}). Dari segi arsitektur jaringan, penelitian ini mengevaluasi CRDT berbasis \textit{client-server}, CRDT berbasis \textit{peer-to-peer}, serta OT berbasis\textit{client-server}. Keterbatasan OT yang diimplementasi pada penelitian ini membutuhkan suatu sumber kebenaran berupa server, sehingga OT berbasis \textit{peer-to-peer} tidak dievaluasi. Penelitian ini menemukan bahwa variasi implementasi CRDT \textit{peer-to-peer} yang diujikan memiliki performa lebih baik untuk sejumlah pengguna $n \leq 8$. Selain itu, \textit{signalling server} pada variasi ini menggunakan \textit{resource} yang minim, sehingga lebih optimal untuk kelompok jaringan yang lebih banyak. Terlepas dari hal tersebut, variasi CRDT \textit{client-server} memiliki pertumbuhan kompleksitas waktu yang lebih rendah terhadap jumlah pengguna dalam satu kelompok jaringan dan penggunaan \textit{resource} sistem klien yang lebih rendah, namun lebih tinggi di servernya. Sehingga variasi CRDT \textit{client-server} dapat dipertimbangkan penggunaannya ketika inisialiasi jaringan \textit{peer-to-peer} yang bermasalah atau untuk perkiraan jumlah pengguna dalam suatu kelompok jaringan yang lebih banyak.\\

\vspace*{0.2cm}

\noindent Kata kunci: \\ WebRTC, WebSocket, \textit{local-first}, waktu nyata, editor kode, \textit{conflict-free replicated data types}, \textit{operational transformation}, \textit{shell} bersama, Yjs, Electron \\

\setstretch{1.4}
\newpage

%
%
%
% Halaman Abstract
%
% @author  Andreas Febrian
% @version 2.1.2
% @edit by Ichlasul Affan
%

\chapter*{ABSTRACT}
\singlespacing

\vspace*{0.2cm}

% Untuk conditional statement pembimbing dua
\def\blank{}

\noindent \begin{tabular}{l l p{11.0cm}}
	Name&: & \penulis \\
	Study Program&: & \studyProgram \\
	Title&: & \judulInggris \\
	Counsellor&: & \pembimbingSatu \\
	\ifx\blank\pembimbingDua
	\else
		\ &\ & \pembimbingDua \\
	\fi
	\ifx\blank\pembimbingTiga
	\else
		\ &\ & \pembimbingTiga \\
	\fi
\end{tabular} \\

\vspace*{0.5cm}

\noindent Real-time collaborative applications are becoming a part of modern human life today. These technologies are used primarily to communicate and increase productivity by reducing time and space barriers. One of the widely used application systems is a document editor that can be used by multiple users simultaneously. Based on this motivation, this research presents a peer-to-peer and local-first collaborative code editor application implemented with WebRTC and CRDT. In addition, the application includes a shared shell that can be run by one user and used by every other user in a network group. There are several variations of backend architecture in the applications compared in this study. In terms of algorithms for maintaining document consistency, two different approaches were evaluated, namely OT (operational transformation) algorithm and CRDT (conflict-free replicated data types) data structure. In terms of network architecture, this study assessed client-server based CRDT, peer-to-peer based CRDT, and client-server based OT. The limitation of OT implemented in this research is that it requires a single source of truth in the form of a server, so peer-to-peer-based OT was not evaluated. This study found that the peer-to-peer based CRDT variation tested performed better for a number of users $n \leq 8$. Moreover, the signaling server in this variation uses minimal resources, making it more optimal for larger network groups. However, the client-server CRDT variation has lower time complexity growth with respect to the number of users in a network group and lower resource usage on the client system, but higher on the server. Therefore, the client-server CRDT variation's usage can be considered when there are problems initializing a peer-to-peer network or the number of users in a network group is much larger than the experiments conducted in this study. \\

\vspace*{0.2cm}

\noindent Key words: \\ WebRTC, WebSocket, local-first, real-time, code editor, conflict-free replicated data types, operational transformation, shared shell, Yjs, Electron \\

\setstretch{1.4}
\newpage


% % Remove space between each chapter in ToC
% \setlength{\cftbeforesecskip}{-1pt}
% \renewcommand{\cftbeforesecskip}{2pt}
\newcommand{\op}{\texttt{op}}

% Remove space between each chapter in ToC
% \newcommand*{\noaddvspace}{\renewcommand*{\addvspace}[0.3]{}}
% \addtocontents{lot}{\protect\noaddvspace}


%
% Daftar isi, gambar, tabel, dan kode
%
\CAPinToC % All entries in ToC will be CAPITALIZED from here on
\phantomsection %hack to make them clickable
\setlength{\cftbeforesecskip}{1.5pt}
\setstretch{1.1}
\tableofcontents
\setstretch{1.4}
\clearpage
\phantomsection %hack to make them clickable
\singlespacing
\listoffigures
\setstretch{1.4}
\clearpage
\phantomsection %hack to make them clickable
\singlespacing
\listoftables
\setstretch{1.4}
\clearpage

%
% Daftar Isi yang Didefinisikan Sendiri (Custom)
% Definisi jenis objek baru dapat dilakukan di uithesis.sty
% Uncomment to use.
%
%\phantomsection %hack to make them clickable
%\addcontentsline{toc}{chapter}{\listmintedcodename}
%\singlespacing
%\listofmintedcode
%\setstretch{1.4}
%\clearpage

%
% Daftar Equation (Persamaan Matematis)
% Uncomment to use.
%
% \phantomsection %hack to make them clickable
% \addcontentsline{toc}{chapter}{\listofequname}
% \singlespacing
% \listofequ
% \setstretch{1.4}
% \clearpage

%
% Daftar Lampiran
% Comment to disable.
%
\phantomsection %hack to make them clickable
\addcontentsline{toc}{chapter}{\listofappendixname}
\singlespacing
\listofappendix
\setstretch{1.4}

% Table of content normal lagi hurufnya
\enableboldchapterintoc

\clearpage

% Jika penomoran romawi selesai di ganjil
%\naiveoddclearchapter
% Jika penomoran romawi selesai di genap
%\naiveevenclearchapter

\noCAPinToC % Revert to original \addcontentsline formatting

%
% Gunakan penomeran Arab (1, 2, 3, ...) setelah bagian ini.
%
\pagenumbering{arabic}
\pagestyle{standard}
% \setlength{\belowcaptionskip}{+2pt}

\setoddevenheader
%-----------------------------------------------------------------------------%
\chapter{\babSatu}
\label{bab:1}
%-----------------------------------------------------------------------------%
Bab ini membahas latar belakang perkembangan teknologi komunikasi waktu nyata, aplikasinya di internet pada saat ini, serta tantangan dan hambatan perkembangan teknologi ini untuk dapat digunakan secara komersial bagi publik. Melalui latar belakang, disusun gagasan pengembangan dan implementasi sederhana sistem aplikasi yang memanfaatkan teknologi tersebut. Ada pula tujuan, manfaat, serta batasan penulisan untuk memberikan konteks visi, misi, dan lingkup pengembangan yang turut disampaikan pada bab ini.

%-----------------------------------------------------------------------------%


\section{Latar Belakang}
\label{sec:latarBelakang}
%-----------------------------------------------------------------------------%
Penggunaan aplikasi yang menghubungkan manusia, baik secara lisan maupun tulisan secara waktu nyata (\textit{real-time}) semakin marak digunakan. Pada situasi pandemi COVID-19 semasa penelitian ini, perusahaan dan institusi pendidikan mengadakan aktivitas jarak jauh dan membutuhkan suatu media komunikasi yang dapat diandalkan. Beberapa aplikasi yang umum digunakan ialah produk-produk Google Workspace, seperti Google Docs, Google Slides, Google Meet, ataupun dari pengembang lain, yakni WhatsApp, LINE, JetBrains Code With Me, Zoom, dan masih banyak lagi. Tidak lepas pula dari permainan video multipemain yang membolehkan pemainnya berinteraksi secara langsung dengan latensi yang cenderung rendah dan terasa seperti waktu nyata.

Teknologi komunikasi waktu nyata atau RTC (\textit{real-time communications}) merupakan sebuah istilah metode telekomunikasi untuk beberapa pengguna yang berinteraksi dengan latensi atau waktu jeda yang relatif rendah terhadap respons pengguna~\citep{arefin2013modified}. Teknologi RTC mulai menjadi fokus penelitian dan penggunaan sejak dikenalkannya teknologi WebSocket pada tahun 2008~\citep{fette2011websocket, reynolds2008web}. Penggunaan WebSocket dimanfaatkan untuk menurunkan latensi dan membuat komunikasi dalam waktu nyata dapat terwujud melalui implementasi yang optimal.

RTC (\textit{real-time communications}) dengan WebSocket diaplikasikan pada salah satu aplikasi penyuntingan dokumen yang umum digunakan yakni Google Docs yang fiturnya dikembangkan pada tahun 2010~\citep{googledocs1}. Google Docs menggunakan metode khusus yaitu OT (\textit{operational transformation}) yang mendukung berbagai kapabilitas kolaborasi~\citep{googledocs2,googledocs3}. Teknologi ini mulai berkembang dan diteliti pada tahun 1989~\citep{Ellis1989}. Sepanjang perkembangannya, ada beberapa isu kesalahan pada metode OT yang terdeteksi dan diselesaikan secara bertahap. Implementasi dari metode ini juga memiliki banyak variasi serta keuntungan dan kerugiannya masing-masing, baik dari aspek memori maupun waktu. Salah satu pengembang aplikasi Google Wave, yang merupakan teknologi pendahulu Google Docs membutuhkan waktu sekitar dua tahun untuk menyelesaikan implementasi dari metode OT ini~\citep{shareJS}. Meskipun membutuhkan waktu lama, Google Docs menjadi salah satu produk editor teks andalan dengan kolaborasi waktu nyata yang memanfaatkan arsitektur \textit{client-server} dan masih digunakan hingga saat ini.

Seiring perkembangan teknologi, pada tahun 2011 WebRTC dikenalkan sebagai protokol dan antarmuka pemrograman aplikasi yang mendukung komunikasi waktu nyata dua arah yang bekerja secara \textit{peer-to-peer}~\citep{dutton2012getting}. WebRTC menyediakan suatu protokol untuk membolehkan suatu pengguna untuk berkomunikasi langsung dengan setiap pengguna lain tanpa melalui server setelah melakukan proses \textit{signalling}. \textit{Signalling} merupakan istilah untuk melakukan inisiasi koneksi kanal data WebRTC melalui sebuah server~\citep{sredojev2015webrtc}. WebRTC dikembangkan dengan tujuan utama untuk melakukan komunikasi waktu nyata dengan data yang lebih besar, seperti media suara dan video~\citep{dutton2012getting}. Muatan ke server juga menjadi lebih ringan karena hanya digunakan untuk \textit{signalling}. Hal ini cenderung meningkatkan skalabilitas dan menyediakan lebih banyak ketersediaan jaringan WebRTC terhadap pengguna bila dibandingkan dengan \textit{client-server}.

WebRTC juga mulai diteliti untuk dapat digunakan dalam berbagai kasus penggunaan, salah satunya untuk penyuntingan dokumen secara berkolaborasi dan dalam waktu nyata. Metode OT yang umumnya digunakan pada arsitektur \textit{client-server} memiliki beberapa properti khusus pada sifat konvergensi hasil akhirnya yang menyebabkan implementasinya pada arsitektur \textit{peer-to-peer} akan lebih sulit~\citep{Sun2017}. Sesuai namanya, OT (\textit{operational transformation}) meresolusi dengan melakukan transformasi terhadap operasi-operasi penyuntingan yang dilakukan terhadap suatu dokumen~\citep{OTOverview1}. Pada arsitektur \textit{client-server}, metode ini mengandalkan suatu server sebagai satu sumber kebenaran data (\textit{single source of truth}) yang akan menyelesaikan resolusi setiap operasi yang masuk dari setiap kliennya.

Perkembangan struktur data yang disebut dengan \textit{conflict-free replicated data type} (CRDT) mulai menjadi alternatif untuk metode resolusi pada arsitektur \textit{peer-to-peer}. CRDT dikenalkan pada tahun 2006 dan mulai secara formal didefinisikan pada tahun 2011~\citep{Shapiro2011}. Struktur data ini digunakan pada komputasi sistem terdistribusi dan tidak membutuhkan koneksi yang selalu tersedia setiap saatnya bagi semua pengguna. Pada CRDT, resolusi dilakukan pada \textit{state} atau kondisi dokumen saat ini dan tidak melalui transformasi dari operasi-operasi penyuntingannya~\citep{CRDToverview1}. CRDT dapat pula digunakan pada arsitektur \textit{client-server}, dengan setiap resolusi diselesaikan pada server, dan perubahan akan di-\textit{broadcast} atau disebarkan pada setiap klien~\citep{Sun2019First}.

Kedua arsitektur, yakni \textit{client-server} dan \textit{peer-to-peer} memiliki banyak keuntungan dan kerugian. Reliabilitas dan sumber daya yang terfokus pada server dapat menjadi terbatas, namun lebih stabil karena performanya yang dipersiapkan oleh pengembang. Di lain sisi, arsitektur \textit{peer-to-peer} dipertimbangkan karena mengurangi waktu \textit{overhead} dalam berkomunikasi antarpenggunanya karena berhubungan langsung dan tidak melalui server. Jaringan \textit{peer-to-peer} dapat bekerja secara optimal saat banyaknya pengguna dalam suatu jaringan kecil~\citep{leibnitz2007peer, maly2003comparison}. Namun sebaliknya, jumlah koneksi dalam jaringan akan meningkat dengan kompleksitas $O(N^2)$ dengan susunan \textit{mesh} bila setiap pengguna terhubung dengan pengguna lainnya. Hal ini menyebabkan komunikasi data yang dilakukan akan meningkat dalam waktu linear dan tidak efisien karena transmisi untuk data yang sama dilakukan untuk semua pengguna. Terlepas dari kelebihan dan kekurangan yang disampaikan, beberapa aplikasi editor kode kolaboratif waktu nyata yang ada saat ini dikembangkan dengan arsitektur tertentu sesuai dengan kebutuhannya. Misalnya terdapat editor kode Atom dengan plugin Teletype, Brackets, dan JetBrains Code With Me yang berbasis \textit{peer-to-peer}, atau pun codecollab, codeshare, dan Google Docs yang berbasis \textit{client-server}.

Editor kode kolaboratif lebih lanjut dapat dikembangkan menjadi suatu \textit{environment} yang membolehkan kolaborasi untuk digunakan dalam pemrograman kompetitif. Pemrograman kompetitif merupakan cabang olahraga pemrograman yang diperlombakan secara individu atau berkelompok untuk mengerjakan soal komputasional dengan batasan waktu dan memori tertentu. Pada pemrograman kompetitif, kode yang diedit umumnya bersifat sebuah berkas tunggal (\textit{single file}), yang dapat dikompilasi atau dijalankan tersendiri. Beberapa bahasa yang umum digunakan antara lain C, C++, Python, dan Java. Penelitian ini akan membahas pengembangan sebuah aplikasi editor kode, akses kompilator, dan \textit{shell} kolaboratif yang mendukung fitur pemrograman bersama dalam waktu nyata untuk setiap pengguna dalam sebuah jaringan. Penelitian ini juga menunjukkan hasil analisis \textit{benchmarking} performa PeerToCP yang menggunakan metode resolusi \textit{operational transformation} berbasis \textit{client-server}, struktur data CRDT berbasis arsitektur \textit{peer-to-peer}, serta struktur data CRDT pula, namun berbasis \textit{client-server}.

%-----------------------------------------------------------------------------%


\section{Pertanyaan Penelitian}
\label{sec:definisiMasalah}
%-----------------------------------------------------------------------------%
Berdasarkan paparan latar belakang dan rumusan masalah yang disampaikan, berikut adalah pertanyaan-pertanyaan yang hendak dijawab dari penelitian.

\begin{enumerate}[noitemsep]
    \item Bagaimana implementasi dari beberapa variasi aplikasi PeerToCP yang menyediakan \textit{real-time collaborative code editor} dan \textit{shared shell}?
    \item Bagaimana perbandingan performa, latensi, dan kebutuhan \textit{resource} sistem pengguna dan server untuk setiap variasi PeerToCP?
    \item Berdasarkan hasil analisis dan evaluasi, apa saja pertimbangan penggunaan sistem dengan variasi tertentu?
\end{enumerate}

%-----------------------------------------------------------------------------%

\section{Batasan Penelitian}
\label{sec:batasanMasalah}
%-----------------------------------------------------------------------------%
Pertanyaan penelitian yang disampaikan pada subbab sebelumnya dibatasi oleh beberapa batasan penelitian. Pembatasan ini untuk memberikan kejelasan cakupan dan jangkauan penelitian yang disampaikan dalam karya ini. Dalam penelitian ini, \textit{user interface} dan \textit{user experience} dari bagian tampilan aplikasi tidak akan diuji dengan metode interaksi manusia dan komputer. PeerToCP masih menggunakan \textit{library}, modul, dan \textit{framework} yang sudah tersedia dengan modifikasi dan penyesuaian seperlunya dalam proses pengembangan sistem aplikasi. Untuk setiap variasi PeerToCP, terdapat beberapa perbedaan fitur dan perilaku minor (tidak signifikan terhadap operasi yang akan dievaluasi) yang diabaikan.

%-----------------------------------------------------------------------------%


\section{Tujuan Penelitian}
\label{sec:tujuan}
%-----------------------------------------------------------------------------%
Terlepas dari batasan dan cakupannya, penelitian ini bertujuan untuk mewujudkan potensi dari teknologi \textit{real-time communications} yakni WebSocket dan WebRTC dalam bentuk aplikasi PeerToCP, sebuah \textit{environment} pemrograman kompetitif kolaboratif \textit{real-time}. Bersama tujuan tersebut, detail implementasi dipaparkan secara detail untuk menerangkan hambatan dan solusi yang diambil dalam pengembangan aplikasi ini. Penelitian ini juga menunjukkan evaluasi perbandingan variasi implementasi dari PeerToCP dengan uji skenario berbagai operasi esensial yang dapat dilakukan.


\section{Manfaat Penelitian}
\label{sec:manfaat}
%-----------------------------------------------------------------------------%
Penelitian ini diharapkan dapat memberikan gambaran umum dan bentuk prototipe dasar beberapa teknologi \textit{real-time communication}, seperti teknologi WebRTC (\textit{Web Real-Time Communication}) dan beberapa metode resolusi sinkronisasi replika data dalam beberapa pengguna dalam sebuah jaringan. Lebih lanjut, aplikasi ini berpotensi untuk dikembangkan lebih lanjut ke tahap produksi dan digunakan secara komersial. Penelitian ini juga didesain untuk menjadi acuan dalam penelitian tolok ukur lain terkait dengan performa arsitektur yang digunakan terhadap beberapa operasi komunikasi data tertentu, terutama dalam bentuk kolaborasi penyuntingan teks dan sinkronisasi data waktu nyata. Berangkat dari tujuan penelitian yang disampaikan sebelumnya pula, beberapa permasalahan dan hambatan pengembangan yang dipaparkan dapat diteliti lebih lanjut untuk membuat alternatif solusi yang lebih optimal terhadap solusi yang disampaikan pada penelitian ini.


\section{Sistematika Penulisan}

Untuk memberikan konteks penelitian yang padu dan terurut, laporan penelitian yang disampaikan dalam karya ini dibagi menjadi enam bagian, antara lain sebagai berikut.

\begin{enumerate}[noitemsep]
    \item Bab I Pendahuluan, memberikan konteks dasar dan pendahuluan dari penelitian, termasuk latar belakang, rumusan masalah yang terdiri dari pertanyaan penelitian dan batasannya, tujuan dan manfaat penelitian, serta sistematika penulisan keseluruhan tulisan.
    \item Bab II Tinjauan Pustaka, menyampaikan dasar-dasar studi dari pustaka yang berhubungan dengan penelitian yang dilakukan. Bab ini juga memberikan pengertian terminologi, teori, dan konsep tertulis terkait.
    \item Bab III Metodologi Penelitian, menerangkan metodologi yang digunakan dalam penelitian ini termasuk tahapan penelitian, desain implementasi, skenario pengujian, dan metrik evaluasi.
    \item Bab IV Implementasi, membahas detail implementasi dari aplikasi PeerToCP dengan berbagai variasinya.
    \item Bab V Hasil dan Pembahasan, menjelaskan hasil evaluasi terhadap pengujian performa yang dilakukan.
    \item Bab VI Penutup, memberikan kesimpulan serta saran akhir untuk perkembangan penelitian selanjutnya.
\end{enumerate}
\clearchapter
%-----------------------------------------------------------------------------%
\chapter{\babDua}
\label{bab:2}
%-----------------------------------------------------------------------------%
Untuk menjawab pertanyaan penelitian yang diuraikan pada Bab I, dibutuhkan dasar pengetahuan yang sesuai. Informasi ini berguna untuk mengetahui potensi pengembangan aplikasi dari berbagai tulisan dan penelitian sebelumnya. Secara umum, bab ini memaparkan mengenai teknologi-teknologi yang terkait dengan pengembangan aplikasi, antara lain WebRTC, CRDT \textit{Conflict-Free Replicated Data Types}, OT (\textit{operational transformation}), dan sifat-sifat pada sebuah editor teks, terutama untuk editor kode. Bab ini juga akan memberikan gambaran mengenai penelitian terkait dan sistem-sistem aplikasi yang sudah pernah dikembangkan sebelumnya. Dalam penelitian ini, teknologi \textit{websocket} menjadi salah satu solusi alternatif yang dapat digunakan untuk PeerToCP berbasis \textit{client-server}.

\section{WebSocket}

WebSocket merupakan protokol komunikasi dengan kanal dua arah, atau biasa dikenal dengan \textit{full-duplex} melalui sebuah koneksi TCP \citep{fette2011websocket}. Protokol ini bersifat \textit{stateful}, yang berarti koneksi antara klien dan server akan terus bertahan hingga salah satu pihak memutuskan hubungannya \citep{pimentel2012communicating} dan berbeda dari protokol HTTP atau HTTPS yang bersifat \textit{stateless} yang digunakan pada sebagian besar halaman web. Pada protokol HTTP atau HTTPS, setiap permintaan atau \textit{request} yang dikirimkan terpisah satu sama lain dan koneksinya berhenti setelah respons diterima \citep{fielding1999hypertext}.

WebSocket memberikan potensi pengiriman data secara langsung untuk setiap data baru yang tersedia tanpa menginisiasi koneksi baru. Pada arsitektur perangkat lunak, teknologi ini dapat dimanfaatkan untuk membuat sebuah pola penerbit-pelanggan atau \textit{publisher-subscriber design pattern} \citep{ganaputra2015asynchronous}. Pada pola ini, klien dapat melakukan permintaan berlangganan ke suatu server dan menjalin hubungan \textit{WebSocket}. Sementara server akan senantiasa memberikan arus data terus menerus setelah adanya pembaharuan kepada setiap klien yang berlangganan. Selain itu, protokol RPC (\textit{Remote Procedure Call}) juga dapat diterapkan di atas \textit{WebSocket}. RPC merupakan istilah pada sistem terdistribusi yang bekerja seperti pemanggilan fungsi pada sebuah \textit{service} atau layanan aplikasi dengan parameter tertentu \citep{srinivasan1995rpc}. Pada \textit{WebSocket}, setiap pemanggilan \textit{request} RPC menggunakan kanal komunikasi yang sama dan sudah tersedia, sehingga memberikan latensi yang jauh lebih optimal tanpa biaya inisiasi awal tambahan.

\section{WebRTC}
WebRTC merupakan sebuah teknologi web pada browser dan perangkat telepon yang membolehkan koneksi langsung dengan basis arsitektur peer-to-peer dalam transmisi datanya. WebRTC bukan hanya API (\textit{Application Programming Interface}), namun juga termasuk protokol yang telah didefinisikan pada W3C (World Wide Web Consortium) dan IETF (Internet Engineering Task Force). WebRTC dipublikasikan sebagai teknologi \textit{open-source} oleh Google pada Mei 2011, dan API-nya secara \textit{native} dikembangkan dalam bahasa JavaScript. Terdapat beberapa komponen utama dalam WebRTC, antara lain ialah sebagai berikut.

\subsection{RTCPeerConnection}

RTCPeerConnection merupakan sebuah antarmuka yang merepresentasikan sebuah koneksi antara suatu komputer dan \textit{peer} lainnya dalam suatu jaringan \textit{peer-to-peer}. Dalam sebuah jaringan WebRTC dengan skema \textit{full mesh}, suatu komputer pada sebuah jaringan WebRTC dengan $N$ \textit{peers} akan memiliki $N-1$ RTCPeerConnection dengan setiap komputer lainnya dalam jaringan. Terdapat pula skema-skema lain yang mengoptimisasi bentuk jaringan \textit{peer-to-peer} ini dengan keuntungan dan kerugian tertentu.
    
\subsection{MediaStream}

MediaStream merepresentasikan sebuah \textit{stream} atau arus multimedia berupa suara atau video. Pada umumnya sebuah MediaStream dapat mengandung satu atau lebih MediaStreamTrack yang merupakan \textit{track} audio atau video. MediaStreamTrack dapat ditambahkan pada RTCPeerConnection yang nantinya dapat diterima oleh ujung lain dari koneksi tersebut. MediaStream akan menggunakan protokol UDP secara bawaan.
\subsection{RTCDataChannel}
RTCDataChannel merupakan kanal data yang digunakan untuk mentransmisikan data apa saja dalam sebuah RTCPeerConnection. Sebuah koneksi tersebut dapat memiliki hingga $65534$ RTCDataChannel. Berbeda dengan MediaStream, RTCDataChannel umumnya digunakan sebagai kanal untuk membagikan pesan teks atau biner antar klien. API WebRTC juga menyediakan dua jenis mode pengiriman, yakni sebagai berikut.

\begin{itemize}[noitemsep]
    \item Pengiriman pesan berurutan dan \textit{reliable}, yang konsep pengirimannya sama dengan data yang ditransmisikan dengan protokol TCP (Transmission Control Protocol). Potensi penggunaannya dapat digunakan untuk pengiriman pesan atau berkas.
    \item Pengiriman pesan yang tidak harus berurutan dan memperbolehkan kekurangan pesan yang ekuivalen dengan UDP (User Datagram Protocol). Potensi penggunannya bisa untuk permainan, pengendalian perangkat jarak jauh, serta banyak lagi karena memngurangi biaya komputasi \textit{overhead} untuk setiap transmisi datanya, sehingga mode ini bertransmisi dengan lebih cepat.
    \item Pengiriman pesan \textit{partial reliable} dengan protokol SCTP (\textit{Stream Control Transmission Protocol}) yang dapat didefinisikan waktu maksimal \textit{timeout} dan maksimal transmisi ulangnya, urutan dari pesan juga dapat dikonfigurasi.
\end{itemize}

\subsection{\textit{Signalling Server}}

Sebelum memulai sebuah koneksi antar \textit{peer} dan transmisi media dilakukan, suatu \textit{peer} hendaknya mengetahui informasi semua atau sebagian \textit{peer} lain yang terdapat dalam jaringan tersebut, tergantung dari skema arsitektur \textit{peer-to-peer} yang diimplementasi. \textit{Signalling server} bertindak sebagai sebuah server yang mengelola koneksi antar perangkat, namun tidak mengelola lalu lintas media transmisi data itu sendiri. server ini hanya sebagai perantara yang memberikan kondisi suatu jaringan dan menandakan \textit{peer} mana saja yang masih terhubung dalam jaringan tersebut. Beberapa tanggung jawab yang dilakukan oleh server ini ialah antara lain sebagai berikut.
 
\begin{itemize}[noitemsep]
    \item Membolehkan sebuah \textit{peer} untuk menemukan \textit{peer} lain di dalam jaringan.
    \item Mengarahkan pembuatan koneksi untuk \textit{peer} baru yang masuk ke dalam sebuah jaringan WebRTC.
    \item Mengulang, mematikan, atau melakukan \textit{reset} sebuah koneksi bila diperlukan.
\end{itemize}

Proses \textit{signalling} ini tidak didefinisikan caranya secara langsung dan memiliki banyak metode alternatif. Terdapat beberapa protokol yang bisa digunakan untuk melakukan \textit{signalling}, antara lain XMPP (Extensible Messaging and Presence Protocol), XHR (XML HTTP Request), dan masih banyak lagi. Salah satu yang umum digunakan lainnya adalah SIP (Session Initiation Protocol) over WebSocket. WebSocket merupakan salah satu protokol yang membuka sebuah sesi jaringan antara klien dan server, dan membolehkan pesan untuk dikirim dari dua arah tanpa memerlukan inisiasi koneksi untuk setiap transmisi datanya. Pesan ini dapat berupa data teks atau pun biner.

\subsection{SDP (\textit{Session Description Protocol})}

Untuk memulai sebuah jaringan, terdapat sebuah objek informasi yang disebut Session Description Protokol yang akan ditawarkan kepada \textit{peer} yang baru masuk ke dalam jaringan WebRTC dan berisi informasi-informasi tertentu mengenai \textit{peer} yang menawarkan tersebut. Misalnya berupa alamat URL, jenis media yang ditransmisikan, \textit{codec}, dan masih banyak lagi. SDP akan dikirimkan kepada signalling server. Setelah \textit{peer} yang ditawarkan menerima, maka \textit{peer} yang ditawarkan tersebut akan memberikan SDP-nya kepada \textit{peer} yang menawarkan, sehingga sebuah jaringan WebRTC akan terjalin. Kandidat yang dapat menerima SDP ini dideskripsikan melalui sebuah ICE Candidate, yaitu sekumpulan rute yang dapat dilalui oleh sebuah \textit{peer} untuk dapat meraih \textit{peer} lain secara langsung. Di dalam SDP, terdapat deskripsi ICE Candidates ini. Dalam beberapa kasus, ICE Candidates akan dikirimkan melalui \textit{signalling server} dengan metode \textit{trickle}, yakni terpisah dari SDP dan ditambahkan satu per satu saat ada ICE Candidate baru yang didapat dari STUN server.

\subsection{ICE (Interactive Connectivity Establishment)}

Karena sistem alamat di Internet kebanyakan masih menggunakan protokol IPv4 yang secara praktis tidak dapat memenuhi semua kebutuhan penetapan alamat sehingga setiap perangkat memiliki alamat IP yang berbeda, perangkat yang digunakan pada suatu jaringan umumnya berada di belakang lapisan NAT (\textit{Network Address Translation}). Mekanisme ini memetakan alamat IP Privat menjadi IP Publik atau sebaliknya saat paket data bertransmisi dalam jaringan. NAT pada umumnya diimplementasikan pada sebuah jaringan dalam lingkup kecil, misalnya pada Wi-fi rumah atau instansi tertentu.

Pada WebRTC, untuk mengetahui alamat \textit{peer} satu sama lain dibutuhkan suatu protokol yang disebut ICE (Interactive Connectivity Establishment). Server ICE akan mengembalikan ICE Candidate yang mendeskripsikan rute dan protokol yang harus diambil untuk mencapai suatu \textit{peer} tertentu. Terdapat dua jenis server untuk ICE, yaitu STUN (\textit{Session Traversal Utilities for NAT}) and TURN (\textit{Traversal Using Relays around NAT}).

\subsection{STUN (\textit{Session Traversal Utilities for NAT})}

Server STUN merupakan server yang mengembalikan alamat IP publik terhadap \textit{peer} yang menghubungi server itu sendiri, jenis NAT yang digunakan, dan \textit{port} NAT yang diasosiasikan dengan \textit{peer} tersebut. Pengembang umumnya menggunakan server STUN publik yang dapat digunakan secara bebas, salah satunya milik Google.

\subsection{TURN (\textit{Traversal Using Relays around NAT})}

Apabila koneksi langsung antar-\textit{peer} gagal dilakukan, maka TURN server berguna sebagai server perantara atau \textit{relay server} yang meneruskan koneksi. Hal ini bisa terjadi karena adanya \textit{firewall} yang diletakkan pada bagian mana saja dari jaringan yang memotong hubungan langsung lalu lintas dari WebRTC. TURN merupakan sebuah protokol untuk meneruskan lalu lintas jaringan yang tidak bisa dilakukan secara langsung tersebut. Sebuah TURN server memiliki public IP address yang dapat diakses oleh kedua \textit{peer}, sehingga TURN Server ini dapat bertindak sebagai sebuah jembatan dalam transmisi media antara dua buah \textit{peer} dalam sebuah jaringan WebRTC.

\section{Editor Kode Kolaboratif}

Editor kode merupakan sebuah peralatan atau aplikasi yang digunakan oleh seorang programmer untuk mengembangkan kodenya. Fungsi-fungsi dasar editor kode yang membedakannya dengan editor biasa misalnya sorotan \textit{syntax}, indentasi otomatis, dan penyocokan tanda kurung otomatis. Selain yang disebutkan, masih ada fungsi-fungsi lain yang tidak ada pada editor teks biasa. Dalam penelitian ini, semua operasi yang digunakan dalam editor kode dapat direduksi secara tidak langsung menjadi operasi-operasi pada editor teks biasa (\textit{plain text editor}). Pada \textit{plain text editor}, setiap karakter pada teks tidak mengandung informasi tambahan. Perhatikan ilustrasi pada Gambar \ref{fig:2:richtext} yang menunjukkan perubahan \textit{styling} yang dapat dilakukan pada \textit{rich text editor}. Operasi semacam ilustrasi tersebut diasumsikan tidak dapat dilakukan pula pada editor kode, karena setiap karakter dianggap tidak menyimpan informasi tambahan.

\begin{figure}
    \centering
    \includegraphics[scale=0.8]{assets/skripsi/richtext.jpg}
    \caption{Diagram Contoh Perubahan pada \textit{Rich Text Editor}}
    \label{fig:2:richtext}
\end{figure}

Pada editor kode atau teks yang bersifat kolaboratif, terdapat beberapa karakteristik yang harus dipenuhi, antara lain sebagai berikut.

\begin{itemize} [noitemsep]
    \item Setiap pengguna memiliki replikat dari dokumen teks.
    \item Pengguna bebas melakukan penyuntingan secara bersamaan tanpa ada larangan tertentu.
    \item Operasi lokal akan diterapkan langsung pada replikat lokalnya tanpa ada jeda.
    \item Operasi yang dilakukan oleh seorang pengguna akan dipropagasi pada setiap pengguna lain secara langsung dengan latensi minimal, sehingga sifat kolaborasi waktu nyata dapat terwujud.
    \item Terdapat beberapa algoritma yang akan mewujudkan konsistensi \textit{state} atau keadaan dokumen pada setiap replikatnya. Setiap operasi yang dilakukan oleh setiap pengguna akan menghasilkan dokumen yang konvergen dan identik.
    \item Operasi yang dilakukan bersifat komutatif, yang berarti terlepas dari urutan diterapkannya operasi pada suatu dokumen, hasilnya akan tetap sama melalui algoritma yang mewujudkan konsistensi ini.
\end{itemize}

\subsection{OT (\textit{Operational Transformation})}
\label{subsec:OT}

Salah satu tantangan dalam menciptakan suatu sistem terdistribusi adalah untuk memiliki suatu basis atau struktur data yang nilainya konsisten untuk setiap klien dalam sistem tersebut. Salah satu struktur data yang menjadi fokus penelitian adalah dokumen \textit{plain text}. Metode OT (\textit{Operational Transformation}) dikembangkan dengan motivasi bagi setiap pengguna dalam suatu sistem terdistribusi dapat memiliki dokumen yang sama untuk setiap perubahan yang terjadi \citep{Sun1998}. Dalam algoritma dasar OT, operasi yang digunakan adalah \texttt{insert(pos, c)}. Operasi tersebut memasukkan sebuah karakter $c$ pada indeks $\texttt{pos}$ dan setiap karakter yang posisi awalnya berada $\geq \texttt{pos}$ akan digeser ke indeks selanjutnya. Ada pula operasi \texttt{delete(pos)} atau menghapus sebuah karakter pada indeks \texttt{pos} \citep{OTOverview1}. Unit operasi seperti \texttt{insert} dan \texttt{delete} ini merupakan unit dasar dari OT \citep{OTOverview1}.

OT dibuat untuk menyelesaikan konflik operasi yang dapat terjadi tanpa mengetahui urutan terjadi antar setiap kliennya. OT secara garis besar bekerja melalui sebuah fungsi $T$, yang mentransformasikan dan menyesuaikan parameter suatu operasi $\op$ yang akan dilakukan pada suatu dokumen, berdasarkan operasi-operasi sebelumnya yang telah diterapkan pada dokumen tersebut. Terdapat dua sifat yang harus dipenuhi oleh suatu algoritma OT untuk bekerja, antara lain sebagai berikut \citep{crdtLecture, OTOverview1}.

\begin{itemize}
    \item CP1/TP1 (\textit{Convergent Property} 1 atau \textit{Transformation Property} 1), yaitu $\op_1 \circ T(\op_2, \op_1) \equiv \op_2 \circ T(\op_1, \op_2)$.
    \item CP2/TP2 (\textit{Convergent Property} 2 atau \textit{Transformation Property} 2), yaitu $T(\op_{3},\op_{1}\circ T(\op_{2},\op_{1}))=T(\op_{3},\op_{2}\circ T(\op_{1},\op_{2}))$.
\end{itemize}

\begin{figure}[h]
    \centering
    \caption{Diagram Ilustrasi OT}
    \label{fig:OTschema}
\end{figure}

\subsection{CRDT (\textit{Conflict-Free Replicated Data Type})}

CRDT merupakan suatu tipe data abstrak, yang berarti semantiknya didefinisikan dari kumpulan nilai dan operasi. Implementasi dari CRDT untuk setiap operasinya bisa berbeda-beda, tapi menghasilkan \textit{behavior} yang sama. CRDT didesain untuk disimpan pada setiap node atau \textit{peer} dalam sebuah jaringan. Oleh karena itu, implementasi CRDT yang efisien terhadap memori dan waktu juga menjadi pertimbangan dalam menggunakan struktur data ini. struktur data ini memiliki karakteristik:

\begin{itemize}[noitemsep]
    \item setiap replika bisa dimodifikasi tanpa berkoordinasi dengan replika lain; kemudian
    \item bila setiap replika dilakukan operasi yang sama tanpa memerhatikan urutannya, maka semuanya akan menghasilkan \textit{state} atau keadaan akhir yang sama.
\end{itemize}

Salah satu dari contoh CRDT yang sederhana ialah \textit{unordered set} atau \textit{himpunan tak berurut}. Pada tipe data tersebut, setiap \textit{peer} dapat melakukan operasi \texttt{insert(v)}, yaitu menambahkan suatu elemen \texttt{v} ke dalam \textit{set} atau himpunan. Selanjutnya, ada pula operasi \texttt{erase(v)} yang akan menghapus elemen \texttt{v} dalam himpunan bila ada. Dalam penelitian ini, tipe data CRDT digunakan untuk mengolah proses pengolahan teks, sehingga operasi-operasi yang terkait dengan CRDT yakni serupa dengan yang disampaikan dengan operasi pada bagian \ref{subsec:OT}, yakni \texttt{insert(pos, c)} serta \texttt{delete(pos)}.

\section{Penelitian Terkait}

\section{Aplikasi dan \textit{Framework} Terkait}

Terdapat beberapa aplikasi dan \textit{library} yang terkait dalam pengembangan sistem aplikasi PeerToCP yang dibahas dalam penelitian ini. Berbagai \textit{framework}, \textit{library}, dan sistem modul ini merupakan hasil penelitian oleh para pengembang sebelumnya. 

\subsection{Electron}

Electron merupakan salah satu \textit{framework} aplikasi desktop yang melibatkan HTML, CSS, dan JavaScript. Bagian belakang atau \textit{backend} dari Electron berjalan dengan lingkungan \textit{runtime} Node.js. Node.js merupakan bahasa yang menggunakan \textit{syntax} yang serupa dengan Javascript dan dapat dikompilasi melalui kopmilator yang disebut V8 engine. Bagian tampilan atau \textit{frontend} dari Electron memanfaatkan aplikasi \textit{Chromium} yang dapat mengolah bahasa \textit{markup} web, seperti HTML (\textit{HyperText Markup Language}), CSS (\textit{Cascading Style Sheets}), serta JavaScript.

Salah satu keuntungan menggunakan Electron adalah aplikasinya yang bersifat \textit{cross-platform} atau dapat berjalan di beragam sistem operasi, seperti Windows, GNU/Linux, atau MacOS. Keuntungan lainnya ialah karena bersifat aplikasi desktop, Electron dapat mengakses berbagai macam fungsi antar muka sistem operasi, seperti memanggil subproses pada kernel dan menulis berkas. Karena \textit{frontend}-nya yang menggunakan bahasa web pula, aplikasi yang dibuat dengan Electron cenderung lebih mudah untuk dipindahkan dan diadaptasi dengan fungsi terbatas pada web.


\subsection{Yjs}

Yjs merupakan sebuah \textit{framework} library yang mengimplementasi CRDT yang disebut dengan YATA (\textit{Yet Another Transformation Approach}) \citep{Nicolaescu2016yjs}. 

\subsubsection{y-docs}

\subsubsection{y-protocols}

\subsubsection{y-websocket}

\subsubsection{y-webrtc}

\subsection{Codemirror}

Codemirror merupakan komponen \textit{frontend} editor kode yang dapat diolah oleh peramban web. Codemirror menyediakan banyak ekstensi, aksesibilitas tinggi, serta dukungan untuk berbagai macam bahasa pemrograman. Codemirror berguna untuk menampilkan editor kode dan memiliki ekstensi yang menghubungkannya dengan Yjs dan sudah diuji oleh pengembang Codemirror.

\subsection{Xterm.js}

Xterm.js merupakan salah satu komponen \textit{frontend} yang menampilkan terminal melalui bahasa yang dapat diolah oleh web. Xterm.js memiliki antarmuka yang bisa menerima dan meneruskan data dari peramban (\textit{browser}) yang dapat dihubungkan dengan sebuah proses berjalan pada kernel. Xterm.js dikembangkan tanpa memerlukan dependensi, sehingga cocok digunakan untuk pengembangan sistem yang membutuhkan tampilan \textit{shell} secara instan.

\subsection{Node-pty}

Node-pty merupakan \textit{library} Node.js yang memberikan antarmuka untuk melakukan \textit{fork} proses dengan deskriptor berkas \textit{pseudoterminal}. Node-pty mengizinkan adanya aliran data untuk baca dan tulis dengan proses berjalan pada kernel. Node-pty berguna untuk menjalankan berkas hasil kompilasi yang bersifat CLI (\textit{Command Line Interface}) yang tidak memiliki tampilan grafik untuk pengguna. Node-pty juga bersifat \textit{cross-platform} mendukung sistem operasi Windows, GNU/Linux, dan MacOS.

\subsection{ShareJS}




\clearchapter
%-----------------------------------------------------------------------------%
\chapter{\babTiga}
\label{bab:3} 

Bab ini secara umum memaparkan tentang metodologi penelitian yang ditempuh dalam mengembangkan sistem PeerToCP. Lebih lanjut, akan mencakup mengenai pendekatan, rincian tahapan, yang termasuk desain sistem, serta metode evaluasi sistem.

\section{Pendekatan dan Tahapan Penelitian}

\begin{figure}
    \centering
    \includegraphics[scale=0.7]{assets/skripsi/MetodePenelitian.pdf}
    \caption{Bagan Alur Penelitian}
    \label{fig:my_label}
\end{figure}

\section{Desain Sistem}

\subsection{Kode dan Kompilasi}

Proses kompilasi merupakan proses mengonversi kode dari bahasa dengan level yang lebih tinggi dan dapat dimengerti oleh manusia menjadi kode biner yang dapat dimengerti oleh mesin. Pada penelitian ini, selain editor kode yang bersifat kolaboratif, proses kompilasi kode tunggal juga hendaknya dapat dilakukan oleh salah satu pengguna. Proses kompilasi ini membutuhkan kompilator yang terpasang pada suatu sistem operasi. 

\section{Metode dan Skenario Evaluasi}

\clearchapter
%-----------------------------------------------------------------------------%
\chapter{\babEmpat}
\label{bab:4}

Bab ini menjelaskan detail arsitektur dan implementasinya. Perbedaan dari performa aplikasi ini memberikan kesempatan untuk melakukan optimisasi desain sistem terdistribusi ke depannya. Detail parameter dan cara melakukan tolok ukur performa terhadap variasi aplikasi PeerToCP juga akan dipaparkan lebih lanjut dalam bab ini pada subbab~\ref{sec:desain_evaluasi} tentang desain dan parameter evaluasi. Bab ini juga akan memberikan penjelasan singkat serta alasan pemilihan beberapa teknologi, termasuk \textit{library} dan \textit{modul} yang digunakan dalam implementasi aplikasi PeerToCP. Sebagai konteks, bagian awal dari bab ini menjelaskan mengenai teknologi yang digunakan untuk implementasi, diikuti dengan desain sistem serta detail sudut pandang pengguna terhadap \textit{usecase} aplikasi.

\section{\textit{Library} dan \textit{Framework} Terkait}

Terdapat beberapa aplikasi dan \textit{library} yang terkait dalam pengembangan sistem aplikasi PeerToCP yang dibahas dalam penelitian ini. Berbagai \textit{framework}, \textit{library}, dan sistem modul ini merupakan hasil penelitian oleh para pengembang sebelumnya. Pemilihan penggunaan untuk setiap teknologi ini dipertimbangkan dengan alasan tertentu, salah satunya adalah Electron, yang menjadi basis pengembangan aplikasi \textit{desktop}.

Electron merupakan salah satu \textit{framework} aplikasi \textit{desktop} yang melibatkan HTML, CSS, dan JavaScript. Bagian belakang atau \textit{backend} dari Electron berjalan dengan lingkungan \textit{runtime} Node.js~\citep{kredpattanakul2018transforming, miglanielectron}. Node.js merupakan bahasa yang menggunakan \textit{syntax} yang serupa dengan Javascript dan dapat dikompilasi melalui kompilator yang disebut V8 engine~\citep{tilkov2010node}. Bagian tampilan atau \textit{frontend} dari Electron memanfaatkan aplikasi \textit{Chromium} yang dapat mengolah bahasa \textit{markup} web, seperti HTML (\textit{HyperText Markup Language}), CSS (\textit{Cascading Style Sheets}), serta JavaScript.

Salah satu keuntungan menggunakan Electron adalah aplikasinya yang bersifat \textit{cross-platform} atau dapat berjalan di beragam sistem operasi, seperti Windows, GNU/Linux, atau MacOS. Keuntungan lainnya ialah karena bersifat aplikasi desktop, Electron dapat mengakses berbagai macam fungsi antar muka sistem operasi, seperti memanggil subproses pada sistem dan menulis berkas. Karena \textit{frontend}-nya yang menggunakan bahasa web pula, aplikasi yang dibuat dengan Electron cenderung lebih mudah untuk dipindahkan dan diadaptasi dengan fungsi terbatas pada web. Selain \textit{Electron}, masih terdapat beberapa alternatif \textit{desktop-based framework} lain seperti Qt yang berbasis C++ dan Tauri yang berbasis Rust. Pemilihan Electron dipilih karena beberapa \textit{library} \textit{operational transformation}, \textit{CRDT}, \textit{WebSocket}, dan \textit{WebRTC} yang umum digunakan sudah tersedia implementasinya dalam JavaScript dan dapat digunakan melalui Node.js.

Untuk memenuhi kebutuhan komponen editor kode sebagai media interaksi pengguna dengan sistem pada \textit{frontend}, digunakan \textit{Codemirror}. Codemirror merupakan komponen \textit{frontend} editor kode yang dapat diolah oleh peramban web. Codemirror menyediakan banyak ekstensi, aksesibilitas tinggi, serta dukungan untuk berbagai macam bahasa pemrograman. Codemirror berguna untuk menampilkan editor kode dan memiliki ekstensi yang menghubungkannya dengan Yjs, sebuah library CRDT dan sudah diuji oleh pengembang Codemirror.

Yjs sendiri merupakan sebuah \textit{framework library} yang mengimplementasi CRDT yang disebut dengan YATA (\textit{Yet Another Transformation Approach})~\citep{Nicolaescu2016yjs}. Yjs terdiri dari beberapa bagian, yaitu YDocs yang merupakan bagian utama implementasi berbagai struktur data untuk CRDT. Dalam penelitian ini, digunakan tiga struktur data CRDT yang abstraksinya berbeda. YText merupakan variasi CRDT untuk operasi-operasi pada \textit{text editor}, serta YMap dan YArray yang dikombinasikan untuk menyimpan \textit{shell} dan riwayatnya. Yjs sendiri merupakan \textit{library} yang tidak terpaku pada sebuah arsitektur. Terdapat dua \textit{provider} jaringan yang dapat berintegrasi dengan YDocs, yaitu YWebRTC dan YWebSocket. Kedua provider ini masing-masing mengintegrasikannya dengan jaringan \textit{full-mesh peer-to-peer} dan \textit{client-server} secara berturut-turut. Yjs memiliki banyak pengembang aktif dari komunitas dan hingga kini masih di-\textit{maintain} dan dikembangkan, sehingga \textit{library} ini dipilih untuk penelitian ini.

Untuk variasi \textit{operational transformation} dari PeerToCP, penelitian ini memanfaatkan ekstensi \textit{collaborative editing} dari CodeMirror yaitu @codemirror/collab. Penyimpanan \textit{shell} diimplementasi tanpa \textit{operational transformation} karena dapat disimpan menggunakan \textit{array} yang bersifat \textit{grow-only}. Secara khusus, operasi penghapusan atau melakukan \textit{backspace} pada masukan \textit{shell} dapat dianggap sebagai menambahkan tiga karakter "$\texttt{\char`\\ b \char`\\ b}$" (tanpa tanda petik) atau ekuivalen dengan memindahkan ke \textit{cursor} ke kiri, \textit{whitespace}, dan memindahkan \textit{cursor} ke kiri lagi tanpa menghapus karakter secara harafiah. Operasi menggeser \textit{cursor} ke kiri tidak dapat dilakukan pada shell secara \textit{default}. Di lain sisi, \textit{provider} jaringan untuk arsitektur \textit{clien-server} dikembangkan dalam penelitian ini menggunakan \textit{library} rpc-websockets yang dimodifikasi sehingga dapat dimanfaatkan untuk melakukan \textit{broadcast}, \textit{specific-messaging} ke klien tertentu, serta fungsionalitas pemanggilan RPC (\textit{Remote-Procedure Call}) berbentuk \textit{promise} secara \textit{asynchronous} dan \textit{non-blocking}.

Untuk memenuhi komponen penjalanan program yang dapat diakses oleh setiap klien dalam jaringan, dibutuhkan suatu \textit{library} untuk mengakses sistem operasi untuk melakukan kompilasi terhadap kode. Kompilasi merupakan proses mengonversi kode dari bahasa dengan level yang lebih tinggi dan dapat dimengerti oleh manusia menjadi kode biner yang dapat dimengerti oleh mesin~\citep{aho1985compilers}. Pada penelitian ini, selain editor kode yang bersifat kolaboratif, proses kompilasi kode tunggal juga hendaknya dapat dilakukan oleh salah satu pengguna. Proses kompilasi ini membutuhkan kompilator yang terpasang pada suatu sistem operasi. Pada Node.js, salah satu \textit{library} yang dapat digunakan untuk mengaksesnya ialah Node-pty.

Node-pty merupakan \textit{library} Node.js yang memberikan antarmuka untuk melakukan \textit{fork} proses dengan deskriptor berkas \textit{pseudoterminal}. Node-pty mengizinkan adanya aliran data untuk baca dan tulis dengan proses berjalan pada kernel. Node-pty berguna untuk menjalankan berkas hasil kompilasi yang bersifat CLI (\textit{Command Line Interface}) yang tidak memiliki tampilan grafik untuk pengguna. Node-pty dipilih karena banyak digunakan dan bersifat \textit{cross-platform} mendukung sistem operasi Windows, GNU/Linux, dan MacOS. Aplikasi ini juga membutuhkan bagian \textit{frontend} untuk menampilkannya, dan digunakan Xterm.js. \textit{Library} ini merupakan salah satu komponen yang menampilkan terminal melalui bahasa yang dapat diolah oleh web. Xterm.js memiliki antarmuka yang bisa menerima dan meneruskan data dari peramban (\textit{browser}) yang dapat dihubungkan dengan sebuah proses berjalan pada sistem. Xterm.js dikembangkan tanpa memerlukan dependensi, sehingga dipilih dalam pengembangan sistem ini.

\section{Desain Sistem}

Aplikasi PeerToCP didesain sebagai sebuah aplikasi \textit{desktop-based}, yang berarti dijalankan tidak melalui browser \textit{web}. Pilihan ini dikonsiderasi karena untuk mempermudah akses secara \textit{offline}, sehingga tidak dibutuhkan koneksi internet untuk mengakses aplikasi. Selain itu, fitur kompilasi pada suatu \textit{peer} memerlukan akses \textit{system-call} yang tidak disediakan pada API web-browser~\citep{v8, spidermonkey}. Hal ini ditetapkan agar \textit{script} yang dijalankan pada mesin browser tidak dapat menyerang komputer secara langsung. Berikut ialah \textit{diagram activity} yang menunjukkan garis besar penggunaan aplikasi.

\begin{figure}
    \centering
    \includegraphics[scale=0.5]{assets/skripsi/Activity_Diagram}
    \caption{\textit{Activity Diagram} Alur Penggunaan Secara \textit{High Level}}
    \label{fig:activity}
\end{figure}

Saat pengguna membuka aplikasi, pengguna akan diarahkan untuk masuk ke ruangan awal secara \textit{default}. Pengguna dapat melakukan operasi-operasi penyuntingan pada dokumen dan sinkronisasi dilakukan secara terus-menerus dengan \textit{publish-subscribe design pattern} sehingga memberikan respons tanpa perlu mengecek atau \textit{polling} secara terus menerus saat terjadi \textit{update} yang terjadi antarklien. \textit{Request} atau permintaan kompilasi dapat diajukan kepada klien mana pun pada jaringan, termasuk permintaan untuk klien ini sendiri. Aplikasi PeerToCP akan mencoba mengirimkan pesan kepada klien yang ditentukan tanpa mengabari tanpa intervesi dari klien lain dalam jaringan. Apabila permintaan berhasil diterima, sistem pada klien yang diminta akan melakukan proses kompilasi dan memasukkan \textit{shell} program berjalan dengan \texttt{id} tertentu ke daftar \textit{shell} yang dapat diakses ke setiap klien dalam jaringan. Daftar \textit{shell} dan kontennya ini disimpan dalam bentuk \textit{object} pada \textit{JavaScript}.

Pada Gambar~\ref{fig:activity}, perilaku sinkronisasi \textit{shell-shell} program dan dokumen dilakukan tergantung dengan variasi implementasi dari program. Pada arsitektur \textit{client-server}, sistem aplikasi pengguna A akan berhubungan dan melakukan sinkronisasi dengan server. Sementara pada arsitektur \textit{peer-to-peer}, sistem aplikasi pengguna A akan berhubungan langsung dan melakukan sinkronisasi dengan \textit{peer} atau klien lain. Selain itu, permintaan dan transmisi pesan hasil kompilasi dilakukan melalui pengiriman pesan secara langsung pada arsitektur \textit{peer-to-peer}, namun harus melalui perantara server pada arsitektur \textit{client-server}. Detail implementasi dan arsitektur detail aplikasi akan dirincikan pada Bab~\ref{bab:4} Implementasi. Sistem yang telah dikembangkan kemudian akan dilakukan evaluasi secara objektif berdasarkan aspek-aspek tertentu yang menrepresentasikan performa dan skalabilitas aplikasi.

\section{CRDT (\textit{Conflict-Free Replicated Data Type}) Berbasis Peer-To-Peer}

\begin{figure}
    \centering
    \includegraphics[scale=0.6]{assets/skripsi/Arsitektur_WebRTC_CRDT}
    \caption{Arsitektur yang Menggunakan WebRTC dengan \textit{WebSocket Signalling} dan CRDT}
\end{figure}

\section{Metode \textit{Operational Transformation} Berbasis \textit{Client-Server}}

\begin{figure}
    \centering
    \includegraphics[scale=0.42]{assets/skripsi/Arsitektur_WebSocket_OT}
    \caption{Arsitektur yang Menggunakan WebSocket dan \textit{Operational Transformation}}
\end{figure}

Untuk suatu dokumen yang hanya menyimpan teks biasa (\textit{plain text}), operasi \textit{operational transformation} dapat diimplementasi dengan sederhana. Algoritma bekerja dengan menyimpan sebuah \textit{array} perubahan lokal yang belum diketahui oleh server. Perubahan-perubahan lokal beserta versi dokumen yang dapat disimpulkan dari banyaknya perubahan yang sudah dilakukan dari awal dokumen akan dikirimkan secara berkala dengan mekanisme RPC over WebSocket. Mekanisme ini pada dasarnya mengizinkan alur \textit{non-blocking} dalam menunggu balasan sebelum mengirimkan percobaan pengiriman perubahan lainnya. Apabila versi dasar dari server lebih baru daripada versi lokal, maka pengiriman perubahan akan ditolak, dan server akan memberikan respons kepada klien tersebut untuk melakukan \textit{rebase} atau menambahkan \textit{update} yang terdapat di server terlebih dahulu sebelum dapat mengirim percobaan pengiriman lagi.

Perubahan \textit{remote} yang diterima dari server akan ditransformasikan satu sama lain dengan perubahan lokal yang belum diketahui oleh server. Perubahan lokal akan ditimpa dengan perubahan yang sudah ditransformasikan dengan perubahan \textit{remote}. Percobaan pengiriman yang dikirim selanjutnya akan mengikuti perubahan yang sudah ditransformasikan ini. \textit{Operational transformation} hanya terjadi di lokal dan perubahan akan diperbarui secara seri, sehingga setiap dokumen dalam jaringan memiliki replika yang berujung identik dan konvergen.

\section{CRDT (\textit{Conflict-Free Replicated Data Type}) Berbasis \textit{Client-Server}}

\begin{figure}
    \centering
    \includegraphics[scale=0.42]{assets/skripsi/Arsitektur_WebSocket_CRDT}
    \caption{Arsitektur yang Menggunakan WebSocket dan \textit{CRDT}}
\end{figure}


\section{Desain Evaluasi}
\label{sec:desain_evaluasi}
\clearchapter
%-----------------------------------------------------------------------------%
\chapter{\babLima}
\label{bab:5}

Bab ini membahas mengenai hasil dan analisis dari evaluasi yang telah dilaksanakan berdasarkan skenario-skenario yang telah disusun pada bab sebelumnya. Pembahasan analisis akan dibagikan berdasarkan aspek-aspek yang telah disusun untuk memberikan gambaran yang lebih jelas mengenai performa program dan \textit{resource} atau sumber daya yang dibutuhkan untuk mencapai performa tersebut. Selain itu, bab ini juga akan membahas pertimbangan variasi dari PeerToCP yang lebih baik digunakan dan faktor-faktor yang memengaruhi pertimbangan tersebut. Melalui pertimbangan tersebut, kelemahan dari sistem aplikasi turut disampaikan untuk perkembangan ke depannya.

\section{Aspek \textit{Local-First} dan \textit{Correctness}}

Kedua aspek ini tidak dapat dipisahkan satu sama lain, pada skenario pertama dan kedua, aspek \textit{local-first} diuji dengan mensimulasikan proses pemutusan koneksi secara acak pada klien, sementara data terus diubah oleh setiap pengguna lokal. Berdasarkan eksperimen secara langsung, proses perubahan ini terjadi secara \textit{local-first}, yang berarti perubahan lokal dapat terus dilakukan dan langsung diterapkan meskipun klien sedang tidak berada dalam jaringan, dan ketika terjadi proses masuk kembali ke jaringan, data akan diperbaharui secara sesuai dengan perubahan yang dilakukan.

Berdasarkan eksperimen yang dilakukan, setiap variasi dari PeerToCP menghasilkan nilai cacahan yang sama untuk setiap skenarionya, yang berarti setiap dokumen berada pada kondisi atau \textit{state} akhir yang sama. Namun pada skenario pertama dengan 8 klien pada variasi Operational Transformation yang berbasis \textit{Client-Server}, dapat terjadi pemutusan hubungan \textit{disconnection} yang tidak dapat terhubung kembali karena \textit{bandwidth} koneksi internet yang berada di luar batas \textit{environment} pengujian.

Skenario dengan jumlah klien ini diulang hingga tiga kali eksperimen, dan dalam setiap eksperimen tersebut, satu hingga dua klien tidak dapat terhubung kembali. Perhatikan bahwa pada eksperimen, waktu pemutusan hubungan dilakukan secara acak dalam periode durasi yang sama untuk setiap kliennya seperti yang dijelaskan pada bab sebelumnya. Hal ini ditunjukkan secara lebih detail pada grafik transmisi data yang masuk melalui jaringan pada klien pertama dan klien kedua untuk setiap variasi aplikasi dalam skenario pertama ini.

\begin{figure}
 \centering
 \includegraphics[width=13cm]{./assets/skripsi/p2cp_cell_2_output_26}
 \caption{Grafik Perbandingan Jaringan pada Klien Pertama dan Klien Kedua untuk $n = 8$}
 \label{fig:2-26}
\end{figure}

Pada saat operasi \textit{update} dilakukan tepat setelah koneksi terhubung kembali, ukuran \textit{update} cenderung lebih besar karena sudah terkumpul dari beberapa operasi yang terjadi selama klien sedang berada di luar jaringan. Ketika \textit{update} ini dilakukan, server sedang dalam keadaan berbagai klien lain yang hendak mengirimkan \textit{update}. Karena hal tersebut, algoritma \textit{operational transformation} yang hanya membolehkan suatu \textit{update} untuk dilakukan apabila versi terbaru yang sama sudah dimiliki oleh klien akan meminta klien untuk melakukan \textit{update}. Sementara tumpukan \textit{push update} berukuran besar terus dikirimkan hal ini akan memenuhi \textit{bandwidth} server, yang dapat dilihat pada grafik berikut.

Perhatikan bahwa pada Gambar~\ref{fig:2-26}, menunjukkan klien pertama yang mengalami pemutusan jaringan pada detik ke-72 dan penghubungan kembali pada detik ke-102, terjadi pengiriman data yang cukup besar selama kurang lebih 20 detik. Begitu pula dengan klien kedua yang mengalami pemutusan jaringan pada sekitar detik ke-60 dan penghubungan kembali pada detik ke-90. Sesaat setelah suatu klien terhubung kembali ke suatu jaringan, akan terjadi \textit{spike} pada jaringan yang cukup besar. Pada jumlah pengguna yang cukup banyak dalam satu dokumen, pemutusan dan penghubungan jaringan yang secara sengaja dilakukan pada waktu yang serupa dapat menyebabkan ketidakandalan pada sistem ini. Grafik untuk server ditunjukkan pada gambar berikut.

\begin{figure}
 \centering
 \includegraphics[width=15cm]{./assets/skripsi/p2cp_cell_2_output_23}
 \caption{Grafik Perbandingan Penerimaan Data pada Setiap Variasi Server PeerToCP}
 \label{fig:2-23}
\end{figure}

Lalu lintas jaringan pada server variasi \textit{operational transformation} dengan arsitektur \textit{client-server} dapat mencapai lebih dari 800000 \textit{kilobits}/detik atau setara dengan 100 \textit{megabytes}/detik, sementara untuk skenario yang serupa, transmisi data pada server dengan variasi CRDT berarsitektur \textit{client-server} lebih hemat sekitar 950 kali. Melalui Gambar~\ref{fig:2-26}, diketahui bahwa mean transmisi jaringannya sekitar 80 kali lebih sedikit dibandingkan CRDT \textit{peer-to-peer} serta 950 kali lebih sedikit dibandingkan CRDT \textit{client-server}.

Pada skenario pertama ini, terlihat bahwa variasi PeerToCP yang menggunakan CRDT lebih dapat diandalkan karena paralelitas \textit{update} yang dilakukan terhadap dokumen. Faktor-faktor lain yang memengaruhi variasi CRDT yang lebih baik juga tidak menutup kemungkinan dari segi implementasi \textit{provider websocket} yang digunakan, teknik kompresi dan \textit{encoding} yang diterapkan, serta efek bola salju yang ditimbulkan akibat \textit{update} berukuran besar yang terus dikirimkan menjadi tertumpuk dan selalu terlambat dibandingkan \textit{update} lain dan menyangkut aspek selanjutnya yang akan dibahas, yaitu \textit{Scalability} dan \textit{Responsiveness}.

\section{Aspek \textit{Scalability} dan \textit{Responsiveness}}

\begin{table}[H]
 \centering

 \caption{Deskripsi Statistik Aktivitas dan \textit{Resource Server} pada Skenario Pertama}
\begin{tabular}{|cc|r|r|r|}
\hline
\multicolumn{2}{|c|}{$\boldsymbol{n}$} & \multicolumn{1}{c|}{\textbf{2}} & \multicolumn{1}{c|}{\textbf{4}} & \multicolumn{1}{c|}{\textbf{8}} \\ \hline
\multicolumn{1}{|c|}{\multirow{4}{*}{\textbf{CRDT Peer-To-Peer}}} & CPU & 0.01 & 0.02 & 0.10 \\ \cline{2-5}
\multicolumn{1}{|c|}{} & RAM & 0.67 & -0.03 & 0.40 \\ \cline{2-5}
\multicolumn{1}{|c|}{} & Net In & 10.72 & 14.81 & 60.72 \\ \cline{2-5}
\multicolumn{1}{|c|}{} & Net Out & -10.24 & -15.79 & -83.11 \\ \hline
\multicolumn{1}{|c|}{\multirow{4}{*}{\textbf{CRDT Client Server}}} & CPU & 0.93 & 1.29 & 1.68 \\ \cline{2-5}
\multicolumn{1}{|c|}{} & RAM & 5.01 & 6.83 & 9.54 \\ \cline{2-5}
\multicolumn{1}{|c|}{} & Net In & 73.26 & 176.64 & 324.19 \\ \cline{2-5}
\multicolumn{1}{|c|}{} & Net Out & -71.79 & -193.90 & -391.66 \\ \hline
\multicolumn{1}{|c|}{\multirow{4}{*}{\textbf{OT Client Server}}} & CPU & 4.85 & 14.18 & 24.33 \\ \cline{2-5}
\multicolumn{1}{|c|}{} & RAM & 30.92 & 41.88 & 69.99 \\ \cline{2-5}
\multicolumn{1}{|c|}{} & Net In & 52318.08 & 167768.04 & 308540.87 \\ \cline{2-5}
\multicolumn{1}{|c|}{} & Net Out & -26494.76 & -84932.31 & -156178.28 \\ \hline
\end{tabular}
\end{table}

\begin{table}[H]
 \centering
 \caption{Deskripsi Statistik Aktivitas dan \textit{Resource} Klien atau \textit{Peers} pada Skenario Pertama}

\begin{tabular}{|cc|r|r|r|}
\hline
\multicolumn{2}{|c|}{$n$} & \multicolumn{1}{c|}{\textbf{2}} & \multicolumn{1}{c|}{\textbf{4}} & \multicolumn{1}{c|}{\textbf{8}} \\ \hline
\multicolumn{1}{|c|}{\multirow{4}{*}{\textbf{CRDT Peer-To-Peer}}} & CPU & 39.1964893 & 60.13947488 & 62.2870496 \\ \cline{2-5}
\multicolumn{1}{|c|}{} & RAM & 16.57856816 & 19.24442935 & 21.30499303 \\ \cline{2-5}
\multicolumn{1}{|c|}{} & Net In & 28.5230253 & 96.16536701 & 229.4788353 \\ \cline{2-5}
\multicolumn{1}{|c|}{} & Net Out & -28.03809012 & -93.97903876 & -223.6266535 \\ \hline
\multicolumn{1}{|c|}{\multirow{4}{*}{\textbf{CRDT Client Server}}} & CPU & 33.96390498 & 57.38616741 & 56.89049428 \\ \cline{2-5}
\multicolumn{1}{|c|}{} & RAM & 14.92823532 & 19.12934751 & 20.94699478 \\ \cline{2-5}
\multicolumn{1}{|c|}{} & Net In & 19.46582482 & 27.68880062 & 30.96869733 \\ \cline{2-5}
\multicolumn{1}{|c|}{} & Net Out & -19.76281471 & -21.21326889 & -18.73440885 \\ \hline
\multicolumn{1}{|c|}{\multirow{4}{*}{\textbf{OT Client Server}}} & CPU & 47.79374975 & 73.29454876 & 65.25737338 \\ \cline{2-5}
\multicolumn{1}{|c|}{} & RAM & 25.9460194 & 31.17265199 & 36.33549154 \\ \cline{2-5}
\multicolumn{1}{|c|}{} & Net In & 62.24127487 & 99.81376692 & 87.42216719 \\ \cline{2-5}
\multicolumn{1}{|c|}{} & Net Out & -12962.04576 & -28872.47276 & -18344.14905 \\ \hline
\end{tabular}
\end{table}

\section{Evaluasi Latensi}

% Please add the following required packages to your document preamble:
% \usepackage{multirow}
\begin{table}[H]
 \centering
 \caption{Deskripsi Statistik Latensi Skenario Ketiga dalam ms}
\begin{tabular}{|c|rrr|rrr|rrr|}
\hline
\multirow{2}{*}{$n$} & \multicolumn{3}{c|}{\textbf{CRDT Peer-To-Peer}} & \multicolumn{3}{c|}{\textbf{CRDT Client Server}} & \multicolumn{3}{c|}{\textbf{OT Client Server}} \\ \cline{2-10}
 & \multicolumn{1}{c|}{Mean} & \multicolumn{1}{c|}{Median} & \multicolumn{1}{c|}{Max} & \multicolumn{1}{c|}{Mean} & \multicolumn{1}{c|}{Median} & \multicolumn{1}{c|}{Max} & \multicolumn{1}{c|}{Mean} & \multicolumn{1}{c|}{Median} & \multicolumn{1}{c|}{Max} \\ \hline
\textbf{2} & \multicolumn{1}{r|}{23.30} & \multicolumn{1}{r|}{21.00} & 76.00 & \multicolumn{1}{r|}{39.50} & \multicolumn{1}{r|}{38.00} & 134.00 & \multicolumn{1}{r|}{87.42} & \multicolumn{1}{r|}{84.00} & 168.00 \\ \hline
\textbf{4} & \multicolumn{1}{r|}{34.00} & \multicolumn{1}{r|}{30.00} & 223.00 & \multicolumn{1}{r|}{46.10} & \multicolumn{1}{r|}{42.00} & 220.00 & \multicolumn{1}{r|}{98.84} & \multicolumn{1}{r|}{86.00} & 1225.00 \\ \hline
\textbf{8} & \multicolumn{1}{r|}{55.56} & \multicolumn{1}{r|}{47.00} & 313.00 & \multicolumn{1}{r|}{63.84} & \multicolumn{1}{r|}{56.00} & 308.00 & \multicolumn{1}{r|}{235.62} & \multicolumn{1}{r|}{151.00} & 2010.00 \\ \hline
\end{tabular}
\end{table}

% Please add the following required packages to your document preamble:
% \usepackage{multirow}
\begin{table}[H]
 \centering
 \caption{Deskripsi Statistik Latensi Skenario Keempat dalam ms}
\begin{tabular}{|c|rrr|rrr|rrr|}
\hline
\multirow{2}{*}{$n$} & \multicolumn{3}{c|}{\textbf{CRDT Peer-To-Peer}} & \multicolumn{3}{c|}{\textbf{CRDT Client Server}} & \multicolumn{3}{c|}{\textbf{OT Client Server}} \\ \cline{2-10}
 & \multicolumn{1}{c|}{Mean} & \multicolumn{1}{c|}{Median} & \multicolumn{1}{c|}{Max} & \multicolumn{1}{c|}{Mean} & \multicolumn{1}{c|}{Median} & \multicolumn{1}{c|}{Max} & \multicolumn{1}{c|}{Mean} & \multicolumn{1}{c|}{Median} & \multicolumn{1}{c|}{Max} \\ \hline
\textbf{2} & \multicolumn{1}{r|}{3.05} & \multicolumn{1}{r|}{17.00} & 19.00 & \multicolumn{1}{r|}{16.75} & \multicolumn{1}{r|}{3.00} & 5.00 & \multicolumn{1}{r|}{31.79} & \multicolumn{1}{r|}{30.00} & 170.00 \\ \hline
\textbf{4} & \multicolumn{1}{r|}{3.13} & \multicolumn{1}{r|}{17.00} & 21.00 & \multicolumn{1}{r|}{16.55} & \multicolumn{1}{r|}{3.00} & 6.00 & \multicolumn{1}{r|}{44.47} & \multicolumn{1}{r|}{32.00} & 309.00 \\ \hline
\textbf{8} & \multicolumn{1}{r|}{3.36} & \multicolumn{1}{r|}{17.00} & 35.00 & \multicolumn{1}{r|}{17.16} & \multicolumn{1}{r|}{3.00} & 19.00 & \multicolumn{1}{r|}{59.86} & \multicolumn{1}{r|}{30.33} & 551.00 \\ \hline
\end{tabular}
\end{table}

\section{Evaluasi Performa}
\clearchapter
%---------------------------------------------------------------
\chapter{\kesimpulan}
\label{bab:6}
%---------------------------------------------------------------
Bab ini memaparkan kesimpulan penelitian dan eksperimen yang telah dilakukan terhadap sistem yang telah dikembangkan. Bab ini memberikan rangkuman singkat dan implikasi dari hasil evaluasi yang telah dianalisis. Selain itu, potensi pengembangan dan eksperimen lebih lanjut yang dapat diteliti di masa yang akan datang juga akan disampaikan guna mengidentifikasi kelemahan dan memberikan kesempatan untuk meneliti topik atau sistem yang lebih optimal .

%---------------------------------------------------------------
\section{Kesimpulan}
\label{sec:kesimpulan}

Penelitian ini dibuat untuk mewujudkan aplikasi PeerToCP, yaitu sebuah editor kode kolaboratif yang menyediakan \textit{shell} bersama yang bekerja dalam waktu nyata. Terdapat beberapa variasi arsitektur dan algoritma yang digunakan dalam aplikasi ini, yaitu algoritma OT (\textit{operational transformation}) dengan arsitektur \textit{client-server}, struktur data CRDT dengan arsitektur \textit{client-server}, serta variasi CRDT lainnya dengan arsitektur \textit{peer-to-peer} berbasis WebRTC.

Penggunaan variasi CRDT yang merupakan pengembangan \textit{operational transformation} dengan struktur data tambahan pada aplikasi PeerToCP menghasilkan performa latensi dan penurunan kebutuhan \textit{resource} yang membuat aplikasi dengan variasi ini lebih dipilih. Untuk mengoptimisasi CRDT lebih lanjut, struktur data \textit{map} atau \textit{dictionary} untuk menyimpan replika \textit{shell} dapat dimodifikasi seperti variasi \textit{operational transformation} dengan memanfaatkan pengetahuan bahwa data hanya akan dimasukkan saja ke ujung \textit{array} tanpa ada proses penghapusan atau pengubahan pada indeks lain di \textit{array}. Variasi \textit{operational transformation} membutuhkan protokol jaringan dan jenis algoritma OT yang lebih baik dari yang saat ini diimplementasikan pada PeerToCP.

Variasi \textit{client-server} dan \textit{peer-to-peer} yang menggunakan WebRTC memiliki kelebihan dan kekurangan masing-masing baik dari segi beban pada server maupun pada pengguna. Berdasarkan skenario-skenario pengujian yang dilakukan pada eksperimen ini, penggunaan untuk skala pengguna yang kecil ($\leq 8$) dan setiap pengguna berada pada jarak yang dekat dapat memanfaatkan versi CRDT \textit{peer-to-peer}. Untuk skala pengguna dalam suatu kelompok yang lebih besar, variasi CRDT \textit{client-server} lebih dipilih karena pertumbuhan transmisi data yang lebih pelan terhadap banyaknya klien dalam suatu kelompok jaringan. Selain itu, variasi \textit{client-server} ini juga lebih dipilih saat beberapa pengguna dalam jaringan kesulitan menginisialisasi koneksi WebRTC karena jaraknya yang berjauhan atau struktur jaringan yang mencegah adanya koneksi WebRTC (terhubung ke \textit{peer} lain secara langsung) terbentuk. Untuk skala pengguna keseluruhan yang lebih besar atau kelompok jaringan yang lebih banyak, variasi CRDT \textit{peer-to-peer} lebih dipilih karena muatan pada servernya yang jauh lebih rendah dan optimal bila dibandingkan dengan arsitektur \textit{client-server}.

%---------------------------------------------------------------
\section{Saran}
\label{sec:saran}

%---------------------------------------------------------------
Berdasarkan hasil penelitian ini, terdapat potensi pengembangan sistem lanjutan untuk membuat sebuah jaringan adaptif tergantung dengan keadaannya dan dapat menjadi salah satu solusi dalam mengoptimisasi layanan yang lebih \textit{reliable} atau dapat diandalkan bagi semua penggunanya. Dari sisi algoritma dalam memastikan kesamaan replika data pada sebuah jaringan, variasi \textit{operational transformation} atau CRDT lain yang lebih optimal dapat digunakan untuk menggantikan yang ada pada variasi PeerToCP saat ini. Variasi ini diharapkan dapat mengoptimalkan latensi dan menurunkan penggunaan sumber daya yang dibutuhkan oleh sistem aplikasi.

Dari aspek jaringan, beberapa teknologi baru seperti HTTP/3 atau versi terbaru dari HTTP menyediakan mekanisme WebTransport yang dapat diteliti lebih lanjut untuk menggantikan protokol WebSocket dalam eksperimen ini. WebTransport direncanakan untuk menyediakan antarmuka pemrograman yang lebih baik dan memiliki semua fitur yang dapat dilakukan oleh WebSocket dengan latensi yang lebih rendah. Selain itu, pengembangan \textit{front-end} dan aspek HCI (\textit{Human-Computer Interaction}) juga dapat ditingkatkan dalam aplikasi ini. Aspek performa dan pengalaman pengguna lebih lanjut dapat diekstensi ke aplikasi web yang dapat diakses tanpa perlu menggunakan aplikasi desktop seperti Electron, namun dengan \textit{drawback} tidak dapat menjadi \textit{host} untuk menyediakan \textit{shell} yang dapat digunakan bersama oleh setiap pengguna dalam jaringan. Beberapa teknologi serta bahasa pemrograman lain yang memiliki performa lebih baik dan penggunaan \textit{resource} lebih ringan dibandingkan Node.js dan Chromium pada Electron juga dapat digunakan untuk menggantikan \textit{framework} aplikasi saat ini. Tauri dan Qt menjadi salah satu teknologi alternatif yang menjadi pertimbangan penulis dalam mengembangkan aplikasi PeerToCP.
\clearchapter

%
% Daftar Pustaka
\CAPinToC % All entries in ToC will be CAPITALIZED from here on
\include{_internals/pustaka}
\clearchapter
\noCAPinToC % Revert to original \addcontentsline formatting

%
% Lampiran
%
\begin{appendix}
	\newcounter{pagetemp}
	\setcounter{pagetemp}{\thepage}
	\include{_internals/markLampiran}
	\clearchapter
	\setcounter{page}{\thepagetemp}
	\stepcounter{page}
	%-----------------------------------------------------------------------------%
\addappendix{Kode dan Implementasi Aplikasi}
\chapter*{Lampiran 1: Kode dan Implementasi Aplikasi}
\label{appendix:implementation}
%-----------------------------------------------------------------------------%
Setiap kode, implementasi aplikasi pengguna, serta eksperimen pengujian pada penelitian ini dapat diakses pada tautan repositori Github sebagai berikut \texttt{\url{https://github.com/hockyy/peertocp}}. Setiap variasi dari PeerToCP dipisahkan berdasarkan \textit{branch}: \texttt{crdt-cs} yang merupakan variasi CRDT dengan arsitektur \textit{client-server}, \texttt{ot-cs} yang merupakan variasi \textit{operational transformation} dengan arsitektur \textit{client-server}, serta \texttt{crdt-p2p} yang merupakan variasi CRDT dengan arsitektur \textit{peer-to-peer}. Panduan untuk menjalankan kembali pengujian dan skenarionya terdapat pada bagian README.md dari branch \texttt{crdt-p2p} yang ditampilkan sebagai \textit{branch} utama dari \textit{repository}. Implementasi dari server dan modifikasi \textit{provider} koneksi yang digunakan pada branch masing-masing dapat diakses pada tautan repositori Github:

\begin{itemize}[nolistsep, noitemsep]
    \item \texttt{crdt-cs}, dapat diakses pada \texttt{\url{https://github.com/hockyy/y-websocket}};
    \item \texttt{crdt-p2p}, dapat diakses pada \texttt{\url{https://github.com/hockyy/y-webrtc}};
    \item \texttt{ot-cs}, dapat diakses pada \texttt{\url{https://github.com/hockyy/peertocp-ot-server}}.
\end{itemize}

Dalam melakukan eksperimen, setiap \textit{instance} diinisialisasi dengan beberapa aplikasi dan pengaturan melalui perintah-perintah sebagai berikut.

\begin{minted}[tabsize=2,breaklines]{bash}
sudo apt install -y git wget screen nginx python-is-python3 g++ make
sudo apt install -y build-essential clang libdbus-1-dev libgtk2.0-dev \
                       libnotify-dev libgnome-keyring-dev libgconf2-dev \
                       libasound2-dev libcap-dev libcups2-dev libxtst-dev \
                       libxss1 libnss3-dev gcc-multilib g++-multilib libasound2 xvfb \
export DISPLAY=192.168.0.5:0.0
curl https://my-netdata.io/kickstart.sh > /tmp/netdata-kickstart.sh && sh /tmp/netdata-kickstart.sh
curl -o- https://raw.githubusercontent.com/nvm-sh/nvm/v0.39.2/install.sh | bash
source ~/.bashrc
nvm install 16
nvm use 16
git clone [URL]

# Menggunakan xvfb karena debian tidak ada desktop
xvfb-run npm start
\end{minted}

Data diambil dan diproses pada perangkat lokal dengan \textit{script} pengunduhan log hasil evaluasi sebagai berikut.

\begin{minted}[tabsize=2,breaklines]{bash}
netd () {
    curl "http://$1:19999/api/v1/data?chart=apps.mem&dimension=node \
          &after=$2&points=0&group=average&gtime=0 \
          &timeout=0&format=csv&options=seconds" \
          > mem-$3.csv
    curl "http://$1:19999/api/v1/data?chart=system.ip \
          &after=$2&points=0&group=average&gtime=0 \
          &timeout=0&format=csv&options=seconds" \
          > network-$3.csv
    curl "http://$1:19999/api/v1/data?chart=apps.cpu&dimension=node \
          &after=$2&points=0&group=average&gtime=0 \
          &timeout=0&format=csv&options=seconds" \
          > cpu-$3.csv
}

scpd () {
    scp -r hocky@$1:~/peertocpnext/out/ ./$2
}
\end{minted}

Setiap server menggunakan NGINX dengan konfigurasi sebagai berikut.

\begin{minted}[tabsize=2,breaklines]{nginx}
location / {
    # First attempt to serve request as file, then
    # as directory, then fall back to displaying a 404.
    # try_files $uri $uri/ =404;
    proxy_pass http://127.0.0.1:3000;

    proxy_http_version  1.1;
    proxy_set_header Upgrade $http_upgrade;
    proxy_set_header Connection "upgrade";
    proxy_set_header Host $http_host;
    proxy_set_header X-Real-IP $remote_addr;
}
\end{minted}

Hasil visualisasi dan interpretasi dari eksperimen yang disampaikan pada penelitian ini dapat diakses mellaui tautan repositori Github sebagai berikut \texttt{\url{https://github.com/hockyy/peertocp-benchmark}}.
\end{appendix}

\end{document}
