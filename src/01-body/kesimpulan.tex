%---------------------------------------------------------------
\chapter{\kesimpulan}
\label{bab:6}
%---------------------------------------------------------------
Bab ini memaparkan kesimpulan penelitian dan eksperimen yang telah dilakukan terhadap sistem yang telah dikembangkan. Bab ini juga memberikan rangkuman singkat tentang implementasi setiap variasi PeerToCP dan implikasi dari hasil evaluasi yang telah dianalisis. Selain itu, potensi pengembangan dan eksperimen lebih lanjut yang dapat diteliti di masa yang akan datang, disampaikan pula guna mengidentifikasi kelemahan dan memberikan kesempatan untuk meneliti topik atau sistem serupa yang lebih optimal.

%---------------------------------------------------------------


\section{Kesimpulan}
\label{sec:kesimpulan}

Penelitian ini dibuat untuk mewujudkan aplikasi PeerToCP, yaitu sebuah editor kode kolaboratif yang menyediakan \textit{shell} bersama dan bekerja dalam waktu nyata. Terdapat beberapa variasi arsitektur dan algoritma untuk menjaga konsistensi data dalam sistem terdistribusi yang digunakan dalam aplikasi ini, yaitu algoritma OT (\textit{operational transformation}) dengan arsitektur \textit{client-server}, struktur data CRDT (\textit{conflict-free replicated data types}) dengan arsitektur \textit{client-server}, dan juga CRDT yang digunakan bersamaan dengan arsitektur \textit{peer-to-peer} berbasis WebRTC.

PeerToCP yang menggunakan \textit{operational transformation} memanfaatkan sifat \textit{transformation property} 1 untuk menjaga konsistensi setiap data pengguna pada suatu jaringan. Setiap perubahan akan direpresentasikan sebagai suatu operasi \textit{update}. Klien dan server akan menyimpan rangkaian operasi ini. Setiap klien dapat melakukan sinkronisasi dokumen dengan klien lain pada jaringan dengan melakukan \textit{pull update} terhadap server. Klien dapat menambahkan atau melakukan \textit{push update} pada server dengan syarat versi terbaru dari server sudah dimiliki oleh klien saat ini. Bila hal tersebut belum dipenuhi, maka klien akan mendapatkan operasi-operasi terbaru pada server, dan menerapkan algoritma \textit{operational transformation} pada semua operasi lokal yang belum diterima oleh server. \textit{Shell} bersama yang disediakan pada aplikasi ini menggunakan konsep yang serupa dengan editor kodenya, namun \textit{operational transformation}-nya dianggap memiliki kompleksitas waktu konstan karena setiap operasinya merupakan operasi \textit{append} pada sebuah \textit{array}.

Selain algoritma OT, terdapat pula variasi CRDT yang pada dasarnya merupakan pengembangan lanjutan dari algoritma OT dengan struktur data tambahan. Dalam penelitian ini, PeerToCP menggunakan \textit{library} Yjs yang mengimplementasikan CRDT YATA. Setiap karakter pada struktur data ini dianggap memiliki \textit{id} yang berbeda. Kemudian YATA menggunakan pendekatan dengan membuat sebuah struktur barisan linear dalam menyimpan karakternya. Posisi sebuah karakter akan direpresentasikan menjadi sebuah hubungan mendahului dan mengikuti karakter lain. Struktur data ini kemudian dapat dihubungkan dengan sebuah \textit{provider} jaringan \textit{client-server} yang menggunakan \textit{library} YWebSocket berbasis WebSocket. Selain itu, terdapat pula variasi CRDT dengan arsitektur jaringan \textit{peer-to-peer} yang menggunakan \textit{library} YWebRTC berbasis WebRTC, serta memanfaatkan WebSocket untuk berkomunikasi dengan \textit{signalling server}-nya. Kedua provider jaringan ini memberikan abstraksi bagi Yjs untuk dapat berkomunikasi dengan pengguna lain dalam jaringan dan melakukan proses sinkronisasi dokumen.

Pada aplikasi PeerToCP, setiap variasi yang dikembangkan telah dibuktikan kebenarannya secara teoritis oleh pengembang kodenya dan secara empiris melalui empat skenario yang dilakukan pada penelitian ini. Variasi utamanya dengan struktur data CRDT \textit{peer-to-peer} menghasilkan performa latensi yang paling baik, serta kebutuhan \textit{resource} yang lebih rendah pada aplikasi penggunanya. Variasi \textit{operational transformation} dari aplikasi mengalami permasalahan transmisi jaringan dengan \textit{bandwidth} terlalu besar karena protokol dan konkurensi \textit{update} yang kurang optimal. Hal ini terjadi saat setiap klien pada suatu kelompok jaringan terus-menerus melakukan \textit{push update} secara bersamaan. Server cenderung tidak dapat menerima operasi \textit{update} pada suatu klien dengan operasi-operasi lokalnya yang sudah tertumpuk. Hal ini dikarenakan server mengutamakan penambahan versi yang dilakukan oleh klien lain dengan data \textit{push update} yang lebih kecil.

Setiap variasi memiliki kelebihan dan kekurangan masing-masing pula baik dari aspek \textit{lightweight} pada server maupun pada aplikasi pengguna. Berdasarkan skenario-skenario pengujian pada eksperimen ini, penggunaan untuk skala $n \leq 8$ dan setiap pengguna berada pada jarak dekat, PeerToCP dengan CRDT \textit{peer-to-peer} berbasis WebRTC merupakan pilihan optimal. Alasan lainnya ialah karena muatan pada servernya jauh lebih rendah bila dibandingkan dengan arsitektur \textit{client-server}. Pada versi \textit{peer-to-peer} setiap komputasi akan dilakukan oleh pengguna masing-masing, sehingga \textit{signalling server} pada arsitektur ini hanya digunakan untuk menangani pengguna yang terhubung dan terputus pada suatu jaringan.

Pada kasus penggunaan pada kelompok jaringan yang lebih besar, variasi CRDT \textit{client-server} dapat dipertimbangkan karena pertumbuhan transmisi data terhadap banyaknya klien yang lebih rendah. Selain itu, variasi ini juga dapat dipertimbangkan saat beberapa pengguna dalam jaringan kesulitan melanjutkan koneksi WebRTC karena jaraknya berjauhan atau adanya mekanisme jaringan yang menggagalkan koneksi untuk terhubung ke \textit{peer} lain secara langsung.

%---------------------------------------------------------------


\section{Saran}
\label{sec:saran}

%---------------------------------------------------------------
Berdasarkan hasil penelitian ini, terdapat potensi pengembangan sistem lanjutan untuk membuat sebuah jaringan adaptif tergantung dari keadaannya. Potensi ini dapat menjadi salah satu solusi dalam mengoptimisasi layanan yang lebih \textit{reliable} atau dapat diandalkan bagi semua penggunanya. Dari sisi algoritma dalam memastikan kesamaan replika data pada sebuah jaringan, variasi \textit{operational transformation} atau CRDT lain yang lebih optimal dapat diteliti lebih lanjut untuk menggantikan yang ada pada variasi PeerToCP saat ini. Variasi ini diharapkan dapat mengoptimalkan latensi dan menurunkan penggunaan sumber daya yang dibutuhkan oleh sistem aplikasi. Salah satu metode yang berpotensi untuk mengoptimisasi variasi CRDT lebih lanjut ialah memodifikasi struktur data CRDT \textit{map} atau \textit{dictionary} yang digunakan untuk menyimpan replika \textit{shell}. Struktur data CRDT ini seharusnya dapat memanfaatkan sifat bahwa data hanya akan dimasukkan ke ujung \textit{array} saja tanpa ada proses penghapusan atau pengubahan pada indeks lain.

Selain algoritma, aspek jaringan dan teknologi web juga memiliki kesempatan pengembangan yang perlu mendapat perhatian. Beberapa teknologi baru seperti HTTP/3 yang merupakan versi terbaru dari HTTP pada saat penelitian ini dilakukan, menyediakan mekanisme WebTransport yang dapat diteliti lebih lanjut untuk menggantikan protokol WebSocket. WebTransport ditujukan untuk menyediakan antarmuka pemrograman yang lebih baik dan memiliki semua fitur yang dimiliki oleh WebSocket dengan latensi yang lebih rendah. Eksperimen lebih lanjut untuk menguji skalabilitas jumlah pengguna yang lebih besar dapat dilakukan dengan mengubah sistem pengujian tanpa perlu mengakses antarmuka secara \textit{end-to-end} . Selain itu, pengembangan \textit{frontend} dan aspek HCI (\textit{Human-Computer Interaction}) juga dapat ditingkatkan dan diteliti dalam aplikasi PeerToCP.

Aspek performa dan pengalaman pengguna lebih lanjut dapat diperluas ke aplikasi web yang dapat diakses tanpa perlu menggunakan aplikasi desktop seperti Electron, namun dengan \textit{drawback} tidak dapat menjadi \textit{host} untuk menyediakan \textit{shell} yang dapat digunakan bersama oleh setiap pengguna dalam jaringan. Beberapa teknologi serta bahasa pemrograman lain yang memiliki performa lebih baik dan penggunaan \textit{resource} lebih ringan dibandingkan Node.js dan Chromium pada Electron juga dapat diteliti untuk menggantikan \textit{framework} aplikasi saat ini. Tauri dan Qt menjadi salah satu teknologi alternatif yang menjadi pertimbangan penulis dalam mengembangkan aplikasi PeerToCP.
