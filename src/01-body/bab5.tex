%-----------------------------------------------------------------------------%
\chapter{\babLima}
\label{bab:5}

Bab ini membahas mengenai hasil dan analisis dari evaluasi yang telah dilaksanakan berdasarkan skenario-skenario yang telah disusun pada bab sebelumnya. Pembahasan analisis akan dibagikan berdasarkan aspek-aspek yang telah disusun untuk memberikan gambaran yang lebih jelas mengenai performa program dan \textit{resource} atau sumber daya yang dibutuhkan untuk mencapai performa tersebut. Selain itu, bab ini juga akan membahas pertimbangan variasi dari PeerToCP yang lebih baik digunakan dan faktor-faktor yang memengaruhi pertimbangan tersebut. Melalui pertimbangan tersebut, kelemahan dari sistem aplikasi turut disampaikan untuk perkembangan ke depannya.

\section{Aspek \textit{Local-First} dan \textit{Correctness}}

Kedua aspek ini tidak dapat dipisahkan satu sama lain, pada skenario pertama dan kedua, aspek \textit{local-first} diuji dengan mensimulasikan proses pemutusan koneksi secara acak pada klien, sementara data terus diubah oleh setiap pengguna lokal. Berdasarkan eksperimen secara langsung, proses perubahan ini terjadi secara \textit{local-first}, yang berarti perubahan lokal dapat terus dilakukan dan langsung diterapkan meskipun klien sedang tidak berada dalam jaringan, dan ketika terjadi proses masuk kembali ke jaringan, data akan diperbaharui secara sesuai dengan perubahan yang dilakukan.

Berdasarkan eksperimen yang dilakukan, setiap variasi dari PeerToCP menghasilkan nilai cacahan yang sama untuk setiap skenarionya, yang berarti setiap dokumen berada pada kondisi atau \textit{state} akhir yang sama. Namun pada skenario pertama dengan 8 klien pada variasi Operational Transformation yang berbasis \textit{Client-Server}, dapat terjadi pemutusan hubungan \textit{disconnection} yang tidak dapat terhubung kembali karena \textit{bandwidth} koneksi internet yang berada di luar batas \textit{environment} pengujian.

Skenario dengan jumlah klien ini diulang hingga tiga kali eksperimen, dan dalam setiap eksperimen tersebut, satu hingga dua klien tidak dapat terhubung kembali. Perhatikan bahwa pada eksperimen, waktu pemutusan hubungan dilakukan secara acak dalam periode durasi yang sama untuk setiap kliennya seperti yang dijelaskan pada bab sebelumnya. Hal ini ditunjukkan secara lebih detail pada grafik transmisi data yang masuk melalui jaringan pada klien pertama dan klien kedua untuk setiap variasi aplikasi dalam skenario pertama ini.

\begin{figure}
 \centering
 \includegraphics[width=13cm]{./assets/skripsi/benchmark-vis_cell_2_output_23}
 \caption{Grafik Perbandingan Jaringan pada Klien Pertama dan Klien Kedua untuk $n = 8$}
 \label{fig:2-23}
\end{figure}

Pada saat operasi \textit{update} dilakukan tepat setelah koneksi terhubung kembali, ukuran \textit{update} cenderung lebih besar karena sudah terkumpul dari beberapa operasi yang terjadi selama klien sedang berada di luar jaringan. Ketika \textit{update} ini dilakukan, server sedang dalam keadaan berbagai klien lain yang hendak mengirimkan \textit{update}. Karena hal tersebut, algoritma \textit{operational transformation} yang hanya membolehkan suatu \textit{update} untuk dilakukan apabila versi terbaru yang sama sudah dimiliki oleh klien akan meminta klien untuk melakukan \textit{update}. Sementara tumpukan \textit{push update} berukuran besar terus dikirimkan hal ini akan memenuhi \textit{bandwidth} server, yang dapat dilihat pada grafik berikut.

Perhatikan bahwa pada Gambar~\ref{fig:2-23}, menunjukkan klien pertama yang mengalami pemutusan jaringan pada detik ke-72 dan penghubungan kembali pada detik ke-102, terjadi pengiriman data yang cukup besar selama kurang lebih 20 detik. Begitu pula dengan klien kedua yang mengalami pemutusan jaringan pada sekitar detik ke-60 dan penghubungan kembali pada detik ke-90. Sesaat setelah suatu klien terhubung kembali ke suatu jaringan, akan terjadi \textit{spike} pada jaringan yang cukup besar. Pada jumlah pengguna yang cukup banyak dalam satu dokumen, pemutusan dan penghubungan jaringan yang secara sengaja dilakukan pada waktu yang serupa dapat menyebabkan ketidakandalan pada sistem ini. Grafik untuk server ditunjukkan pada gambar berikut.

\begin{figure}
 \centering
 \includegraphics[width=15cm]{./assets/skripsi/benchmark-vis_cell_2_output_24}
 \caption{Grafik Perbandingan Penerimaan Data pada Setiap Variasi Server PeerToCP}
 \label{fig:2-24}
\end{figure}

Lalu lintas jaringan pada server variasi \textit{operational transformation} dengan arsitektur \textit{client-server} dapat mencapai lebih dari 800000 \textit{kilobits}/detik atau setara dengan 100 \textit{megabytes}/detik, sementara untuk skenario yang serupa, transmisi data pada server dengan variasi CRDT berarsitektur \textit{client-server} lebih hemat sekitar 950 kali. Melalui Gambar~\ref{fig:2-24}, diketahui bahwa mean transmisi jaringannya sekitar 80 kali lebih sedikit dibandingkan CRDT \textit{peer-to-peer} serta 950 kali lebih sedikit dibandingkan CRDT \textit{client-server}.

Pada skenario pertama ini, terlihat bahwa variasi PeerToCP yang menggunakan CRDT lebih dapat diandalkan karena paralelitas \textit{update} yang dilakukan terhadap dokumen. Faktor-faktor lain yang memengaruhi variasi CRDT yang lebih baik juga tidak menutup kemungkinan dari segi implementasi \textit{provider websocket} yang digunakan, teknik kompresi dan \textit{encoding} yang diterapkan, serta efek bola salju yang ditimbulkan akibat \textit{update} berukuran besar yang terus dikirimkan menjadi tertumpuk dan selalu terlambat dibandingkan \textit{update} lain dan menyangkut aspek selanjutnya yang akan dibahas, yaitu \textit{Scalability} dan \textit{Responsiveness}.

\section{Aspek \textit{Scalability} dan \textit{Responsiveness}}

Latensi dari teks editor secara khusus diukur secara terus menerus pada skenario ketiga tanpa adanya pemutusan hubungan seperti pada skenario pertama dan kedua. Setiap klien atau \textit{peer} akan diukur waktu selisih penerimaan datanya terhadap klien yang bukan dirinya sendiri dan dicatat latensinya pada waktu data dokumen diterima, kemudian dilakukan rata-rata untuk $n$ klien-klien atau \textit{peers} tersebut tabel~\ref{tab:latency-3} menunjukkan latensi PeerToCP variasi CRDT \textit{peer-to-peer} memiliki latensi yang paling rendah, namun semakin besar nilai $n$, selisih perbedaan CRDT \textit{client-server} dan CRDT \textit{peer-to-peer} semakin mengecil. Sementara pada variasi OT \textit{client-server}, latensinya dua kali lebih besar dari variasi CRDT \textit{client-server}, dengan lonjakan latensi maksimum hingga dua detik saat $n = 8$.

% Please add the following required packages to your document preamble:
% \usepackage{multirow}
\begin{table}[H]
 \centering
 \caption{Statistik Latensi Operasi Nonlokal pada Skenario Ketiga (Teks Editor Bersama) dalam ms}
\begin{tabular}{|c|rrr|rrr|rrr|}
\hline
\multirow{2}{*}{$n$} & \multicolumn{3}{c|}{\textbf{CRDT Peer-To-Peer}} & \multicolumn{3}{c|}{\textbf{CRDT Client Server}} & \multicolumn{3}{c|}{\textbf{OT Client Server}} \\ \cline{2-10}
 & \multicolumn{1}{c|}{Mean} & \multicolumn{1}{c|}{Median} & \multicolumn{1}{c|}{Max} & \multicolumn{1}{c|}{Mean} & \multicolumn{1}{c|}{Median} & \multicolumn{1}{c|}{Max} & \multicolumn{1}{c|}{Mean} & \multicolumn{1}{c|}{Median} & \multicolumn{1}{c|}{Max} \\ \hline
\textbf{2} & \multicolumn{1}{r|}{23.30} & \multicolumn{1}{r|}{21.00} & 76.00 & \multicolumn{1}{r|}{39.50} & \multicolumn{1}{r|}{38.00} & 134.00 & \multicolumn{1}{r|}{87.42} & \multicolumn{1}{r|}{84.00} & 168.00 \\ \hline
\textbf{4} & \multicolumn{1}{r|}{34.00} & \multicolumn{1}{r|}{30.00} & 223.00 & \multicolumn{1}{r|}{46.10} & \multicolumn{1}{r|}{42.00} & 220.00 & \multicolumn{1}{r|}{98.84} & \multicolumn{1}{r|}{86.00} & 1225.00 \\ \hline
\textbf{8} & \multicolumn{1}{r|}{55.56} & \multicolumn{1}{r|}{47.00} & 313.00 & \multicolumn{1}{r|}{63.84} & \multicolumn{1}{r|}{56.00} & 308.00 & \multicolumn{1}{r|}{235.62} & \multicolumn{1}{r|}{151.00} & 2010.00 \\ \hline
\end{tabular}
 \label{tab:latency-3}
\end{table}

Dari segi skalabilitasnya, dengan asumsi pengguna berada dalam jarak yang dekat, misalnya dalam kasus ini setiap \textit{peers} berada dalam satu zona yang sama, yaitu Indonesia. Variasi \textit{peer-to-peer} memberikan performa yang optimal dibandingkan \textit{client-server} untuk jumlah $n$ tertentu. Berdasarkan perhitungan dan proyeksi jumlah pengguna yang bertambah, variasi dari CRDT \textit{client-server} dapat memberikan \textit{responsiveness} yang lebih baik untuk suatu nilai $n$ yang lebih tinggi atau saat \textit{peers} berada pada daerah yang lebih jauh antara satu sama lainnya.

Perbedaan latensi ini dapat disebabkan oleh faktor jaringan dan arsitekturnya atau algoritma yang memastikan bahwa dokumen tersebut sama pada setiap replikanya. Dari analisis algoritma, variasi CRDT akan memberikan latensi melakukan penerapan operasi lokal waktu yang sama. Operasi lokal dalam konteks ini ialahoperasi pada klien tersebut sendiri yang bersifat luring serta \textit{local-first}. dengan  Dalam praktisnya, gambar~\ref{fig:12-5} menunjukkan adanya perbedaan latensi yang tidak terlalu signifikan, yaitu dengan mean hanya sekitar 5ms. Perlu diketahui juga bahwa nilai lonjakan latensi maksimalnya yang berdekatan. Perlu diketahui bahwa skalabilitas latensi algoritma CRDT dan OT tumbuh secara berbeda. Kompleksitas waktu dari OT akan dipengaruhi oleh banyaknya \textit{update} berbeda, sementara kompleksitas waktu CRDT akan dipengaruhi oleh banyaknya karakter atau ukuran dokumen.

\begin{figure}
 \centering
 \includegraphics[width=15cm]{./assets/skripsi/benchmark-vis_cell_12_output_5}
 \caption{Grafik Waktu \textit{Resolve} Operasi Lokal pada Editor Setiap Variasi Klien atau \textit{Peers} PeerToCP}
 \label{fig:12-5}
\end{figure}

Latensi total diperoleh dari waktu penerapan operasi lokal dan nonlokal ditambah dengan waktu transmisi jaringan pada arsitektur yang berbeda-beda. Gambar~\ref{fig:7-5} menunjukkan perbedaan latensi yang cukup besar, serta lonjakan latensi yang terjadi beberapa kali sepanjang skenario ketiga berjalan. Hal ini dapat disebabkan oleh beberapa hal, yakni waktu \textit{resolve} operasi nonlokal dan adanya efek penumpukan \textit{update} yang diilustrasikan pada diagram gambar~\ref{diagram:snowball}.

\begin{figure}
 \centering
 \includegraphics[width=15cm]{./assets/skripsi/benchmark-vis_cell_7_output_5}
 \caption{Grafik Mean Latensi pada Operasi Nonlokal Setiap Variasi Klien atau \textit{Peers} PeerToCP}
 \label{fig:7-5}
\end{figure}

\begin{figure}
 \centering
 \includegraphics[width=15cm]{./assets/skripsi/Snowball}
 \caption{Diagram Ilustrasi Penumpukan \textit{Update}}
 \label{diagram:snowball}
\end{figure}

Pada arsitektur \textit{operational transformation} berbasis \textit{client-server}, \textit{update} dari suatu klien tidak akan diterima oleh server apabila klien tersebut belum memiliki versi terbaru dari server. Penumpukan \textit{update} ini dapat terjadi karena klien tersebut semakin tidak bisa mengirimkan datanya karena semakin besar dan selalu didahului oleh klien lain yang berada dalam jaringan tersebut. Penumpukan ini dapat menyebabkan tidak dapat diperbaruinya salah satu klien dan transmisi data keluar yang besar karena pengiriman \textit{update} besar yang dilakukan terus menerus.

Dalam praktis penggunaan realistisnya, hal ini diperkirakan tidak umum terjadi karena frekuensi pengetikan atau pemasukan karakter tidak dilakukan secara bersamaan dan tidak sebanyak skenario pertama dan kedua. Operasi salin dan tempel secara berkali-kali dapat berpotensi menyebabkan hal ini terjadi karena perubahan dokumen dalam jumlah karakter yang banyak terjadi. Namun dalam praktisnya, operasi tersebut diasumsikan hanya dilakukan dengan jeda frekuensi yang membolehkan setiap klien dalam jaringan memperbarui replikanya sebelum operasi besar lain dilakukan.

Selain operasi penyuntingan dokumen, skenario empat menguji latensi \textit{shell} yang dapat digunakan secara bersama. Secara umum, latensi tersebut digambarkan pada tabel~\ref{tab:latency-4}


% Please add the following required packages to your document preamble:
% \usepackage{multirow}
\begin{table}[H]
 \centering
 \caption{Statistik Latensi Operasi Nonlokal pada Skenario Keempat (\textit{Shell} Bersama) dalam Satuan ms}
\begin{tabular}{|c|rrr|rrr|rrr|}
\hline
\multirow{2}{*}{$n$} & \multicolumn{3}{c|}{\textbf{CRDT Peer-To-Peer}} & \multicolumn{3}{c|}{\textbf{CRDT Client Server}} & \multicolumn{3}{c|}{\textbf{OT Client Server}} \\ \cline{2-10}
 & \multicolumn{1}{c|}{Mean} & \multicolumn{1}{c|}{Median} & \multicolumn{1}{c|}{Max} & \multicolumn{1}{c|}{Mean} & \multicolumn{1}{c|}{Median} & \multicolumn{1}{c|}{Max} & \multicolumn{1}{c|}{Mean} & \multicolumn{1}{c|}{Median} & \multicolumn{1}{c|}{Max} \\ \hline
\textbf{2} & \multicolumn{1}{r|}{3.05} & \multicolumn{1}{r|}{3.00} & 5.00 & \multicolumn{1}{r|}{16.75} & \multicolumn{1}{r|}{17.00} & 19.00 & \multicolumn{1}{r|}{31.79} & \multicolumn{1}{r|}{30.00} & 170.00 \\ \hline
\textbf{4} & \multicolumn{1}{r|}{3.13} & \multicolumn{1}{r|}{3.00} & 6.00 & \multicolumn{1}{r|}{16.55} & \multicolumn{1}{r|}{17.00} & 21.00 & \multicolumn{1}{r|}{44.47} & \multicolumn{1}{r|}{32.00} & 309.00 \\ \hline
\textbf{8} & \multicolumn{1}{r|}{3.36} & \multicolumn{1}{r|}{3.00} & 19.00 & \multicolumn{1}{r|}{17.16} & \multicolumn{1}{r|}{17.00} & 35.00 & \multicolumn{1}{r|}{59.86} & \multicolumn{1}{r|}{30.33} & 551.00 \\ \hline
\end{tabular}
 \label{tab:latency-4}
\end{table}

\begin{table}[H]
 \centering

 \caption{Statistik Aktivitas dan \textit{Resource Server} pada Skenario Pertama}
\begin{tabular}{|cc|r|r|r|}
\hline
\multicolumn{2}{|c|}{$\boldsymbol{n}$} & \multicolumn{1}{c|}{\textbf{2}} & \multicolumn{1}{c|}{\textbf{4}} & \multicolumn{1}{c|}{\textbf{8}} \\ \hline
\multicolumn{1}{|c|}{\multirow{4}{*}{\textbf{CRDT Peer-To-Peer}}} & CPU & 0.01 & 0.02 & 0.10 \\ \cline{2-5}
\multicolumn{1}{|c|}{} & RAM & 0.67 & -0.03 & 0.40 \\ \cline{2-5}
\multicolumn{1}{|c|}{} & Net In & 10.72 & 14.81 & 60.72 \\ \cline{2-5}
\multicolumn{1}{|c|}{} & Net Out & -10.24 & -15.79 & -83.11 \\ \hline
\multicolumn{1}{|c|}{\multirow{4}{*}{\textbf{CRDT Client Server}}} & CPU & 0.93 & 1.29 & 1.68 \\ \cline{2-5}
\multicolumn{1}{|c|}{} & RAM & 5.01 & 6.83 & 9.54 \\ \cline{2-5}
\multicolumn{1}{|c|}{} & Net In & 73.26 & 176.64 & 324.19 \\ \cline{2-5}
\multicolumn{1}{|c|}{} & Net Out & -71.79 & -193.90 & -391.66 \\ \hline
\multicolumn{1}{|c|}{\multirow{4}{*}{\textbf{OT Client Server}}} & CPU & 4.85 & 14.18 & 24.33 \\ \cline{2-5}
\multicolumn{1}{|c|}{} & RAM & 30.92 & 41.88 & 69.99 \\ \cline{2-5}
\multicolumn{1}{|c|}{} & Net In & 52318.08 & 167768.04 & 308540.87 \\ \cline{2-5}
\multicolumn{1}{|c|}{} & Net Out & -26494.76 & -84932.31 & -156178.28 \\ \hline
\end{tabular}
\end{table}

\begin{table}[H]
 \centering
 \caption{Statistik Aktivitas dan \textit{Resource} Klien atau \textit{Peers} pada Skenario Pertama}

\begin{tabular}{|cc|r|r|r|}
\hline
\multicolumn{2}{|c|}{$n$} & \multicolumn{1}{c|}{\textbf{2}} & \multicolumn{1}{c|}{\textbf{4}} & \multicolumn{1}{c|}{\textbf{8}} \\ \hline
\multicolumn{1}{|c|}{\multirow{4}{*}{\textbf{CRDT Peer-To-Peer}}} & CPU & 39.1964893 & 60.13947488 & 62.2870496 \\ \cline{2-5}
\multicolumn{1}{|c|}{} & RAM & 16.57856816 & 19.24442935 & 21.30499303 \\ \cline{2-5}
\multicolumn{1}{|c|}{} & Net In & 28.5230253 & 96.16536701 & 229.4788353 \\ \cline{2-5}
\multicolumn{1}{|c|}{} & Net Out & -28.03809012 & -93.97903876 & -223.6266535 \\ \hline
\multicolumn{1}{|c|}{\multirow{4}{*}{\textbf{CRDT Client Server}}} & CPU & 33.96390498 & 57.38616741 & 56.89049428 \\ \cline{2-5}
\multicolumn{1}{|c|}{} & RAM & 14.92823532 & 19.12934751 & 20.94699478 \\ \cline{2-5}
\multicolumn{1}{|c|}{} & Net In & 19.46582482 & 27.68880062 & 30.96869733 \\ \cline{2-5}
\multicolumn{1}{|c|}{} & Net Out & -19.76281471 & -21.21326889 & -18.73440885 \\ \hline
\multicolumn{1}{|c|}{\multirow{4}{*}{\textbf{OT Client Server}}} & CPU & 47.79374975 & 73.29454876 & 65.25737338 \\ \cline{2-5}
\multicolumn{1}{|c|}{} & RAM & 25.9460194 & 31.17265199 & 36.33549154 \\ \cline{2-5}
\multicolumn{1}{|c|}{} & Net In & 62.24127487 & 99.81376692 & 87.42216719 \\ \cline{2-5}
\multicolumn{1}{|c|}{} & Net Out & -12962.04576 & -28872.47276 & -18344.14905 \\ \hline
\end{tabular}
\end{table}

\section{Evaluasi Latensi}





\section{Evaluasi Performa}