%-----------------------------------------------------------------------------%
\chapter{\babLima}
\label{bab:5}

Bab ini membahas mengenai hasil dan analisis dari evaluasi yang telah dilaksanakan berdasarkan skenario-skenario yang telah disusun pada bab sebelumnya. Pembahasan analisis dibagikan berdasarkan aspek-aspek yang telah disusun untuk memberikan gambaran yang lebih jelas mengenai performa program dan penggunaan \textit{resource} atau sumber daya yang dibutuhkan untuk mencapai performa tersebut. Selain itu, bab ini juga membahas pertimbangan variasi dari PeerToCP yang lebih baik digunakan dan faktor-faktor yang memengaruhi pertimbangan tersebut. Melalui pertimbangan tersebut, kelemahan dari sistem aplikasi turut disampaikan untuk perkembangan ke depannya.

\section{Aspek \textit{Local-First} dan \textit{Correctness}}

Kedua aspek ini tidak dapat dipisahkan satu sama lain. Pada skenario pertama dan kedua, aspek \textit{local-first} diuji dengan menyimulasikan proses pemutusan koneksi secara acak pada klien sementara pengubahan data terus terjadi pada setiap pengguna lokal secara luring. Berdasarkan eksperimen secara langsung, proses perubahan ini terjadi secara \textit{local-first}, yang berarti perubahan lokal dapat terus dilakukan dan langsung diterapkan meskipun klien sedang tidak berada dalam jaringan. Ketika terjadi proses masuk kembali ke jaringan, data akan diperbaharui secara sesuai dengan perubahan yang dilakukan.

Berdasarkan eksperimen yang dilakukan, setiap variasi dari PeerToCP menghasilkan nilai hasil \textit{hash} atau cacahan yang sama untuk setiap skenarionya, yang berarti setiap dokumen berada pada kondisi atau \textit{state} akhir yang sama. Namun pada eksperimen yang dilakukan dalam skenario pertama dengan delapan klien pada variasi \textit{operational transformation} berbasis \textit{client-server}, terjadi pemutusan hubungan \textit{disconnection} yang tidak dapat terhubung kembali. Diperkirakan hal ini terjadi karena ketidakefektifan metode transmisi data pada PeerToCP variasi ini dan menyebabkan kebutuhan \textit{bandwidth} koneksi internet berada di luar batas \textit{environment} pengujian.

Eksperimen skenario pertama variasi OT \textit{client-server} dengan jumlah klien delapan dilakukan kembali hingga tiga kali eksperimen. Dalam setiap eksperimen tersebut, satu hingga dua klien tidak dapat terhubung kembali. Pada eksperimen, waktu pemutusan hubungan dilakukan secara acak dalam periode durasi yang sama untuk setiap kliennya seperti yang dijelaskan pada bab sebelumnya. Grafik lebih detail yang menyatakan transmisi data masuk melalui jaringan pada klien pertama dan klien kedua untuk setiap variasi aplikasi dalam skenario pertama ini ditunjukkan pada gambar grafik~\ref{fig:2-23}.

\begin{figure}
 \centering
 \includegraphics[width=13cm]{./assets/skripsi/benchmark-vis_cell_2_output_23}
 \caption{Grafik Perbandingan Penerimaan Data pada Klien Pertama dan Klien Kedua untuk Skenario Pertama dengan $n = 8$}
 \label{fig:2-23}
\end{figure}

Grafik~\ref{fig:2-23} menunjukkan aktivitas yang cenderung wajar karena ukuran \textit{update} yang diterima tidak terlalu besar. Namun gambar grafik~\ref{fig:2-25} menunjukkan klien pertama yang mengalami pemutusan jaringan pada detik ke-72 dan penghubungan kembali pada detik ke-102, terjadi pengiriman data yang cukup besar selama kurang lebih 20 detik. Begitu pula dengan klien kedua yang mengalami pemutusan jaringan pada sekitar detik ke-60 dan penghubungan kembali pada detik ke-90. Sesaat setelah suatu klien terhubung kembali ke suatu jaringan, terjadi \textit{spike} pada jaringan yang cukup besar. Pada jumlah pengguna yang cukup banyak dalam satu dokumen, pemutusan dan penghubungan jaringan yang secara sengaja dilakukan pada waktu yang serupa dapat menyebabkan ketidakandalan pada sistem ini.

\begin{figure}
 \centering
 \includegraphics[width=15cm]{./assets/skripsi/benchmark-vis_cell_2_output_25}
 \caption{Grafik Perbandingan Pengiriman Data pada Klien Pertama dan Klien Kedua untuk Skenario Pertama dengan $n = 8$}
 \label{fig:2-25}
\end{figure}

Pada saat operasi \textit{update} dilakukan tepat setelah koneksi terhubung kembali, ukuran \textit{update} cenderung lebih besar karena sudah terkumpul dari beberapa operasi yang terjadi selama klien sedang berada di luar jaringan. Ketika \textit{update} ini dilakukan, server sedang dalam keadaan menerima \textit{update} dari berbagai klien lain pula. Variasi algoritma \textit{operational transformation} yang diimplementasikan dalam eksperimen ini hanya memperbolehkan \textit{update} untuk dilakukan apabila versi terbaru di lokal milik klien sama dengan server. Apabila berbeda, maka server akan menolak \textit{update} dan akan meminta klien untuk melakukan \textit{pull update} terlebih dahulu. Sementara hal tersebut terjadi, tumpukan pengiriman \textit{update} berukuran besar terus dikirimkan dan berpotensi memenuhi \textit{bandwidth} server.

\begin{figure}
 \centering
 \includegraphics[width=15cm]{./assets/skripsi/benchmark-vis_cell_2_output_24}
 \caption{Grafik Perbandingan Penerimaan Data pada Server PeerToCP}
 \label{fig:2-24}
\end{figure}

Lalu lintas jaringan pada server variasi \textit{operational transformation} dengan arsitektur \textit{client-server} dapat mencapai lebih dari 800000 \textit{kilobits}/detik atau setara dengan 100 \textit{megabytes}/detik. Melalui gambar~\ref{fig:2-24}, bisa terlihat bahwa rata-rata transmisinya sekitar 950 kali lebih besar dibandingkan variasi CRDT \textit{client-server}. Perhatikan bahwa aktivitas pada server CRDT \textit{peer-to-peer} yang menggunakan protokol WebRTC cenderung minim karena servernya tidak menangani transmisi pertukaran data, melainkan langsung berkomunikasi antar \textit{peers}-nya.

Pada skenario pertama ini, variasi CRDT lebih dapat diandalkan karena paralelisasi \textit{update} yang dilakukan terhadap dokumen. Faktor-faktor lain yang memengaruhi hal tersebut diperkirakan terdapat pada segi implementasi \textit{provider websocket}, teknik kompresi, serta teknik \textit{encoding} yang diterapkan. Selain itu, diperkirakan terjadi  penumpukan \textit{update} akibat operasi-operasi perubahan yang berukuran besar. \textit{Update} tersebut lebih lambat penerimaannya dibandingkan \textit{update} klien lain yang lebih sedikit. Hal ini menyebabkan beberapa klien harus melakukan \textit{pull} berulang kali sebelum dapat meneruskan \textit{update}-nya apabila versi lokal saat ini berbeda dengan yang terdapat pada \textit{server}. Hal ini berakibat pada aspek lain pula, yaitu \textit{scalability} dan \textit{responsiveness}.

\section{Aspek \textit{Scalability} dan \textit{Responsiveness}}

Latensi dari teks editor secara khusus diukur secara terus-menerus pada skenario ketiga tanpa adanya pemutusan hubungan seperti pada skenario pertama dan kedua. Setiap pengguna diukur waktu selisih penerimaan datanya terhadap pengguna yang bukan dirinya sendiri dan dicatat latensinya pada waktu data dokumen diterima. Kemudian latensi tersebut dirata-rata untuk $n$ pengguna pada jaringan. Tabel~\ref{tab:latency-3} merupakan statistik latensi untuk operasi nonlokal yang berarti latensi yang didapatkan dari pengukuran terhadap pengguna lain di dalam jaringan. Tabel tersebut menunjukkan aplikasi PeerToCP variasi CRDT \textit{peer-to-peer} memiliki latensi yang paling rendah. Namun semakin besar nilai $n$, selisih perbedaan CRDT \textit{client-server} dan CRDT \textit{peer-to-peer} semakin mengecil. Sementara pada variasi OT \textit{client-server}, latensinya dua kali lebih besar dari variasi CRDT \textit{client-server}, dengan lonjakan latensi maksimum sekitar dua detik saat $n = 8$.

% Please add the following required packages to your document preamble:
% \usepackage{multirow}
\begin{table}[H]
 \centering

 \caption{Statistik Latensi Operasi Nonlokal pada Skenario Ketiga (Teks Editor Bersama) dalam ms}
 \label{tab:latency-3}
 \begin{tabular}{|c|rrr|rrr|rrr|}
\hline
\multirow{2}{*}{$n$} & \multicolumn{3}{c|}{\textbf{CRDT \textit{Peer-To-Peer}}} & \multicolumn{3}{c|}{\textbf{CRDT \textit{Client-Server}}} & \multicolumn{3}{c|}{\textbf{OT \textit{Client-Server}}} \\ \cline{2-10}
 & \multicolumn{1}{c|}{Mean} & \multicolumn{1}{c|}{Median} & \multicolumn{1}{c|}{Max} & \multicolumn{1}{c|}{Mean} & \multicolumn{1}{c|}{Median} & \multicolumn{1}{c|}{Max} & \multicolumn{1}{c|}{Mean} & \multicolumn{1}{c|}{Median} & \multicolumn{1}{c|}{Max} \\ \hline
\textbf{2} & \multicolumn{1}{r|}{23.30} & \multicolumn{1}{r|}{21.00} & 76.00 & \multicolumn{1}{r|}{39.50} & \multicolumn{1}{r|}{38.00} & 134.00 & \multicolumn{1}{r|}{87.42} & \multicolumn{1}{r|}{84.00} & 168.00 \\ \hline
\textbf{4} & \multicolumn{1}{r|}{34.00} & \multicolumn{1}{r|}{30.00} & 223.00 & \multicolumn{1}{r|}{46.10} & \multicolumn{1}{r|}{42.00} & 220.00 & \multicolumn{1}{r|}{98.84} & \multicolumn{1}{r|}{86.00} & 1225.00 \\ \hline
\textbf{8} & \multicolumn{1}{r|}{55.56} & \multicolumn{1}{r|}{47.00} & 313.00 & \multicolumn{1}{r|}{63.84} & \multicolumn{1}{r|}{56.00} & 308.00 & \multicolumn{1}{r|}{235.62} & \multicolumn{1}{r|}{151.00} & 2010.00 \\ \hline
\end{tabular}
\end{table}

Dari segi skalabilitasnya, dengan asumsi setiap pengguna berada pada jarak yang dekat, misalnya dalam eksperimen ini setiap \textit{peers} berada dalam satu zona yang sama, yaitu Indonesia. Arsitektur \textit{peer-to-peer} memberikan performa yang lebih optimal dibandingkan \textit{client-server} untuk jumlah $n$ tertentu. Berdasarkan perhitungan dan proyeksi jumlah pengguna yang bertambah, variasi dari CRDT \textit{client-server} dapat memberikan \textit{responsiveness} yang lebih baik untuk suatu nilai $n$ yang lebih tinggi atau saat \textit{peers} berada pada daerah yang lebih jauh antara satu dengan lainnya.

Perbedaan latensi ini dapat pula disebabkan algoritma yang memastikan bahwa dokumen tersebut sama pada setiap replikanya. Pada penelitian ini, operasi penerapan perubahan lokal diobservasi untuk mengetahui perbedaan latensi algoritma yang digunakan pada PeerToCP. Operasi lokal dalam konteks ini ialah hanya operasi yang dilakukan pengguna tersebut terhadap dokumen pengguna itu sendiri. Operasi ini bersifat \textit{local-first} dan luring, yang berarti tidak melalui transmisi jaringan. Gambar~\ref{fig:12-5} menunjukkan adanya perbedaan latensi yang tidak terlalu signifikan, yaitu dengan rata-rata sekitar 5ms. Nilai lonjakan latensi maksimalnya juga cenderung tidak signifikan. Lonjakan ini dapat terjadi ketika operasi penambahan atau penghapusan karakter dilakukan pada indeks acak tertentu. Perlu diketahui bahwa skalabilitas latensi algoritma CRDT dan OT tumbuh secara berbeda. Kompleksitas waktu dari OT dipengaruhi oleh banyaknya operasi \textit{update} berbeda, sementara kompleksitas waktu CRDT dipengaruhi oleh banyaknya karakter atau ukuran dokumen.

\begin{figure}
 \centering
 \includegraphics[width=15cm]{./assets/skripsi/benchmark-vis_cell_12_output_5}
 \caption{Grafik Waktu \textit{Resolve} Operasi Lokal pada Editor Aplikasi Pengguna PeerToCP}
 \label{fig:12-5}
\end{figure}

Latensi total diperoleh dari waktu penerapan operasi lokal dan nonlokal ditambah dengan waktu transmisi jaringan pada arsitektur yang berbeda-beda. Gambar~\ref{fig:7-5} menunjukkan perbedaan latensi yang cukup besar, serta lonjakan latensi yang terjadi beberapa kali sepanjang skenario ketiga berjalan. Hal ini dapat disebabkan oleh beberapa hal, yakni waktu \textit{resolve} operasi nonlokal dan adanya efek penumpukan \textit{update} yang diilustrasikan pada diagram gambar~\ref{diagram:snowball}.

\begin{figure}
 \centering
 \includegraphics[width=15cm]{./assets/skripsi/benchmark-vis_cell_7_output_5}
 \caption{Grafik Rata-Rata Latensi pada Operasi Nonlokal Aplikasi Pengguna PeerToCP}
 \label{fig:7-5}
\end{figure}

\begin{figure}
 \centering
 \includegraphics[width=15cm]{./assets/skripsi/Snowball}
 \caption{Diagram Ilustrasi Penumpukan \textit{Update}}
 \label{diagram:snowball}
\end{figure}

Pada arsitektur \textit{operational transformation} berbasis \textit{client-server}, \textit{update} tidak akan diterima oleh server apabila suatu klien belum memiliki versi terbaru dari server. Penumpukan \textit{update} ini terjadi ketika suatu klien tidak bisa mengirimkan datanya karena semakin besar dan selalu didahului oleh klien lain yang berada dalam jaringan tersebut. Hal ini menyebabkan terhambatnya pembaruan salah satu atau beberapa klien dalam jaringan dan mengakibatkan metrik pengiriman data pada klien dan penerimaan data pada server yang relatif jauh lebih besar dibandingkan variasi PeerToCP lainnya.

Dalam praktik penggunaan realistisnya, hal ini diperkirakan tidak umum terjadi karena frekuensi perubahan, pemasukan, dan penghapusan karakter tidak dilakukan secara bersamaan dan sebanyak skenario pertama dan kedua. Operasi salin dan tempel secara berkali-kali dapat berpotensi menyebabkan hal ini terjadi karena perubahan dokumen dalam jumlah karakter yang banyak. Namun, operasi tersebut diasumsikan hanya dilakukan dengan jeda frekuensi yang membolehkan setiap klien dalam jaringan memperbaharui replikanya sebelum operasi besar lain dilakukan.

Selanjutnya, skenario empat menguji latensi \textit{shell} yang dapat digunakan setiap pengguna dalam jaringan secara bersamaan. Secara umum, latensi tersebut dideskripsikan pada tabel~\ref{tab:latency-4}. Berbeda halnya dengan \textit{editor teks} yang waktu penerapan operasinya berdampak sebesar 20--70ms atau lebih. Struktur data berupa \textit{dictionary} sederhana pada \textit{shell} tidak memberikan perbedaan yang signifikan seperti yang digambarkan pada gambar grafik~\ref{fig:13-5}.

% Please add the following required packages to your document preamble:
% \usepackage{multirow}
\begin{table}[H]
 \centering

 \caption{Statistik Latensi Operasi Nonlokal pada Skenario Keempat (\textit{Shell} Bersama) dalam Satuan ms}
 \label{tab:latency-4}
 \begin{tabular}{|c|rrr|rrr|rrr|}
  \hline
\multirow{2}{*}{$n$} & \multicolumn{3}{c|}{\textbf{CRDT \textit{Peer-To-Peer}}} & \multicolumn{3}{c|}{\textbf{CRDT \textit{Client-Server}}} & \multicolumn{3}{c|}{\textbf{OT \textit{Client-Server}}} \\ \cline{2-10}
 & \multicolumn{1}{c|}{Mean} & \multicolumn{1}{c|}{Median} & \multicolumn{1}{c|}{Max} & \multicolumn{1}{c|}{Mean} & \multicolumn{1}{c|}{Median} & \multicolumn{1}{c|}{Max} & \multicolumn{1}{c|}{Mean} & \multicolumn{1}{c|}{Median} & \multicolumn{1}{c|}{Max} \\ \hline
\textbf{2} & \multicolumn{1}{r|}{3.05} & \multicolumn{1}{r|}{3.00} & 5.00 & \multicolumn{1}{r|}{16.75} & \multicolumn{1}{r|}{17.00} & 19.00 & \multicolumn{1}{r|}{31.79} & \multicolumn{1}{r|}{30.00} & 170.00 \\ \hline
\textbf{4} & \multicolumn{1}{r|}{3.13} & \multicolumn{1}{r|}{3.00} & 6.00 & \multicolumn{1}{r|}{16.55} & \multicolumn{1}{r|}{17.00} & 21.00 & \multicolumn{1}{r|}{44.47} & \multicolumn{1}{r|}{32.00} & 309.00 \\ \hline
\textbf{8} & \multicolumn{1}{r|}{3.36} & \multicolumn{1}{r|}{3.00} & 19.00 & \multicolumn{1}{r|}{17.16} & \multicolumn{1}{r|}{17.00} & 35.00 & \multicolumn{1}{r|}{59.86} & \multicolumn{1}{r|}{30.33} & 551.00 \\ \hline
\end{tabular}
\end{table}

\begin{figure}
 \centering
 \includegraphics[width=15cm]{./assets/skripsi/benchmark-vis_cell_13_output_5}
 \caption{Grafik Waktu \textit{Resolve} Operasi Lokal pada \textit{Shell} Aplikasi Pengguna PeerToCP}
 \label{fig:13-5}
\end{figure}

Hal ini dikarenakan operasi penambahan pada \textit{append-only array} dalam \textit{operational transformation} memiliki kompleksitas waktu konstan secara \textit{amortized}. Dari aspek skalabilitas, operasi \textit{shell} relatif hanya berpengaruh terhadap efektivitas dari transmisi data. \textit{Update} operasi \textit{shell} mengalami permasalahan sama seperti pada skenario pertama, kedua, dan ketiga seperti yang ditunjukkan pada gambar grafik~\ref{fig:9-5}.

Faktor lain yang mempengaruhi alasan latensi \textit{operational transformation} lebih lambat dari pada CRDT \textit{client-server} ialah mekanisme komunikasinya. Implementasi CRDT Yjs dilengkapi dengan proses \textit{encoding} dan \textit{decoding} yang membuat transmisinya dalam jaringan untuk data yang besar dapat lebih cepat. Pada CRDT juga operasi \textit{update} yang dilakukan secara berurutan tidak diperlukan. Sementara implementasi variasi \textit{operational transformation} masih menggunakan mekanisme \textit{update} yang sama dengan \textit{editor teks}-nya.

\begin{figure}
 \centering
 \includegraphics[width=15cm]{./assets/skripsi/benchmark-vis_cell_9_output_5}
 \caption{Grafik Rata-Rata Latensi pada Operasi Nonlokal Aplikasi Pengguna PeerToCP}
 \label{fig:9-5}
\end{figure}

Pada eksperimen ini, setiap \textit{peers} pada arsitektur \textit{peer-to-peer} dalam jaringan berada dalam jarak yang cenderung dekat karena berada dalam satu zona yang sama. Latensinya teruji lebih rendah dibandingkan kedua arsitektur \textit{client-server} yang harus melewati server terlebih dahulu. Dalam penelitian ini, untuk $n \leq 8$, variasi \textit{peer-to-peer} masih memberikan performa terbaik dari setiap variasi lainnya dari aspek \textit{responsiveness}. Diperkirakan untuk suatu nilai $n > 8$, atau jarak antar \textit{peers} yang lebih jauh, variasi CRDT dengan arsitektur \textit{client-server} dapat dipertimbangkan sebagai alternatif yang lebih baik.

Dalam beberapa kasus lainnya, CRDT \textit{peer-to-peer} dengan WebRTC dapat mengalami permasalahan antar-\textit{peers} yang tidak dapat saling terhubung. Sehingga, alternatif berupa server penghubung, \textit{relay} atau STUN server dapat digunakan dan latensinya memerlukan eksperimen lebih lanjut untuk dapat dibandingkan. Algoritma \textit{operational transformation} yang lebih baik juga dapat dipertimbangkan, karena secara teori operasi \textit{update}-nya yang konstan dapat memberikan latensi lebih optimal karena memanfaatkan sifat \textit{append-only array} dan fakta bahwa hanya satu host pengguna yang bertanggung jawab untuk meneruskan data interaksinya pada suatu \textit{shell}. Untuk mendapatkan latensi yang lebih baik, struktur data CRDT dapat dimodifikasi lebih lanjut dengan memanfaatkan sifat tersebut.

Sistem dapat ditambahkan pula fitur untuk menentukan arsitektur optimal antara \textit{peer-to-peer} tanpa \textit{relay server} seperti pada eksperimen ini, \textit{peer-to-peer} dengan \textit{relay server} untuk menangani \textit{peer} yang tidak dapat terhubung oleh WebRTC, atau pun \textit{client-server} apabila banyaknya klien dalam suatu kelompok jaringan cukup banyak. Dari segi latensi, pertimbangan hal tersebut dapat ditentukan dengan metrik sederhana seperti \textit{ping} antarpengguna yang dilakukan dari layar belakang aplikasi. Terdapat beberapa faktor lain yang harus dipertimbangkan pula, yakni aspek \textit{lightweight} yang menggambarkan seberapa banyak proses, \textit{bandwidth}, pekerjaan, serta memori yang harus disediakan oleh server dan penggunanya.

\section{Aspek \textit{Lightweight}}

Aspek ini memaparkan seberapa besar \textit{resource} komputasi yang dibutuhkan untuk setiap pengguna dan servernya. Semua metrik RAM pada penelitian ini diukur dengan melakukan normalisasi ke nol sesaat sebelum uji kasus dimulai. Normalisasi ini didasarkan untuk membandingkan perubahan pertumbuhan. Selain itu, dalam beberapa kasus dalam menjalankan klien, aplikasi elektron yang berjalan di atas Chromium memiliki sistem \textit{garbage collection} dan \textit{cache}~\citep{chromium}. Sehingga, pengukuran skenario diasumsikan lebih akurat bila disesuaikan relatif terhadap ukuran memori sesaat sebelum tes dimulai. Pada mulanya, setiap server mulai dengan RAM yang relatif rendah, yakni di bawah 20 MiB. Sementara setiap \textit{peers} mulai dengan RAM yang cenderung lebih tinggi, yakni sekitar 200 MiB. Aplikasi ini membutuhkan memori yang cukup besar karena menggunakan \textit{framework} Electron yang menjalankan peramban web Chromium pada \textit{frontend}-nya.

\begin{table}[H]
 \centering

 \caption{Statistik Rata-Rata Aktivitas dan \textit{Resource} Aplikasi Pengguna pada Skenario Pertama}
 \label{tab:resource-client-1}
 \begin{tabular}{|cc|r|r|r|}
  \hline
\multicolumn{2}{|c|}{$n$} & \multicolumn{1}{c|}{\textbf{2}} & \multicolumn{1}{c|}{\textbf{4}} & \multicolumn{1}{c|}{\textbf{8}} \\ \hline
\multicolumn{1}{|c|}{\multirow{4}{*}{\textbf{CRDT \textit{Peer-To-Peer}}}} & CPU & 39.1964893 & 60.13947488 & 62.2870496 \\ \cline{2-5}
\multicolumn{1}{|c|}{} & RAM & 16.57856816 & 19.24442935 & 21.30499303 \\ \cline{2-5}
\multicolumn{1}{|c|}{} & Net In & 28.5230253 & 96.16536701 & 229.4788353 \\ \cline{2-5}
\multicolumn{1}{|c|}{} & Net Out & -28.03809012 & -93.97903876 & -223.6266535 \\ \hline
\multicolumn{1}{|c|}{\multirow{4}{*}{\textbf{CRDT \textit{Client-Server}}}} & CPU & 33.96390498 & 57.38616741 & 56.89049428 \\ \cline{2-5}
\multicolumn{1}{|c|}{} & RAM & 14.92823532 & 19.12934751 & 20.94699478 \\ \cline{2-5}
\multicolumn{1}{|c|}{} & Net In & 19.46582482 & 27.68880062 & 30.96869733 \\ \cline{2-5}
\multicolumn{1}{|c|}{} & Net Out & -19.76281471 & -21.21326889 & -18.73440885 \\ \hline
\multicolumn{1}{|c|}{\multirow{4}{*}{\textbf{OT \textit{Client-Server}}}} & CPU & 47.79374975 & 73.29454876 & 65.25737338 \\ \cline{2-5}
\multicolumn{1}{|c|}{} & RAM & 25.9460194 & 31.17265199 & 36.33549154 \\ \cline{2-5}
\multicolumn{1}{|c|}{} & Net In & 62.24127487 & 99.81376692 & 87.42216719 \\ \cline{2-5}
\multicolumn{1}{|c|}{} & Net Out & -12962.04576 & -28872.47276 & -18344.14905 \\ \hline
\end{tabular}
\end{table}

Tabel~\ref{tab:resource-client-1} memiliki parameter yang menunjukkan utilitasi CPU dalam satuan persen utilisasi. 100\% utilisasi berarti sistem menggunakan satu \textit{core} vCPU dengan penuh. Selanjutnya terdapat pula sumber daya atau \textit{resource} RAM yang sudah dinormalisasi dalam satuan MiB, diikuti dengan kecepatan penerimaan dan pengiriman data dalam satuan \textit{kilobits} per detik. Penggunaan RAM dan penerimaan data pada variasi CRDT \textit{peer-to-peer} cenderung lebih tinggi untuk $n = 8$, hal ini disebabkan karena pada variasi ini, semakin banyak \textit{peers} pada jaringan, berarti semakin banyak pula replika dokumen yang harus disinkronisasi. Karena \textit{signalling server} pada variasi ini tidak berperan dalam transmisi data CRDT, sehingga setiap \textit{peers} bertanggung jawab untuk saling memberikan informasi satu sama lain dalam memastikan kesamaan informasi pada setiap replikanya. Dari segi banyaknya data yang masuk ke dalam aplikasi, CRDT memastikan bahwa data-data tersebut dapat digunakan secara langsung dan setiap data yang dikirimkan akan digunakan tanpa ada kasus penolakan \textit{update} seperti pada variasi OT. Sehingga dari segi transmisi data keluar, variasi CRDT dianggap lebih memenuhi aspek \textit{lightweight} dibandingkan variasi OT.

Variasi aplikasi CRDT \textit{client-server} secara relatif memberikan performa yang lebih baik dibandingkan dengan jenis lain. Dari segi utilisasi CPU dalam \textit{environment} pengujian dengan dua core vCPU (utilisasi maksimal 200\%), setiap variasi pada skenario pertama dengan $n \geq 4$ secara rata-rata hanya menggunakan sekitar 1/3 \textit{resource} dari CPU seperti yang terlihat pada gambar grafik~\ref{fig:2-19}. Metrik ini dapat dipertimbangkan dengan tambahan bahwa skenario ini memberikan muatan operasi-operasi yang relatif lebih banyak dari perkiraan aktivitas penggunaan normal. Selain itu, sistem operasi secara optimal akan menentukan proses yang harus diutamakan baik dari segi memori dan prioritas penggunaan CPU saat berjalannya aplikasi. Sehingga tidak menutup kemungkinan, penggunaan \textit{resource} ini dapat lebih besar atau kecil tergantung dengan aktivitas lainnya yang terdapat pada suatu mesin.

\begin{figure}
 \centering
 \includegraphics[width=11cm]{./assets/skripsi/benchmark-vis_cell_2_output_19}
 \caption{Grafik Perbandingan Utilisasi CPU pada Klien Pertama dan Klien Kedua untuk $n = 8$}
 \label{fig:2-19}
\end{figure}

\begin{figure}
 \centering
 \includegraphics[width=11cm]{./assets/skripsi/benchmark-vis_cell_2_output_21}
 \caption{Grafik Perbandingan Penggunaan RAM pada Klien Pertama dan Klien Kedua untuk $n = 8$}
 \label{fig:2-21}
\end{figure}

Penggunaan memori sebelum dinormalisasi rata-rata dimulai dengan menggunakan sekitar 200 MiB, diikuti dengan pertumbuhan penggunaan seperti yang terlihat pada gambar grafik~\ref{fig:2-21}. Dari segi memori, kedua variasi CRDT memberikan pertumbuhan yang lebih optimal dibandingkan variasi OT. Namun, setiap variasinya menggunakan memori yang relatif kecil terhadap kapasitas penuh perangkat \textit{environment} pengujian, yakni hanya sekitar 15\%. Sistem operasi memiliki cara tersendiri dalam mengelola CPU dan RAM yang \textit{idle}. Selama aplikasi tidak menggunakan utilisasi penuh yang mengganggu jalannya aktivitas pengguna dalam berinteraksi dengan komputer, maka setiap variasi PeerToCP dianggap memenuhi aspek \textit{lightweight} dari segi utilisasi CPU dan RAM.

\begin{table}[H]
 \centering
 \caption{Statistik Rata-Rata Aktivitas dan \textit{Resource Server} pada Skenario Pertama}
 \label{tab:resource-server-1}


\begin{tabular}{|cc|r|r|r|}
\hline
\multicolumn{2}{|c|}{$\boldsymbol{n}$} & \multicolumn{1}{c|}{\textbf{2}} & \multicolumn{1}{c|}{\textbf{4}} & \multicolumn{1}{c|}{\textbf{8}} \\ \hline
\multicolumn{1}{|c|}{\multirow{4}{*}{\textbf{CRDT \textit{Peer-To-Peer}}}} & CPU & 0.01 & 0.02 & 0.10 \\ \cline{2-5}
\multicolumn{1}{|c|}{} & RAM & 0.67 & -0.03 & 0.40 \\ \cline{2-5}
\multicolumn{1}{|c|}{} & Net In & 10.72 & 14.81 & 60.72 \\ \cline{2-5}
\multicolumn{1}{|c|}{} & Net Out & -10.24 & -15.79 & -83.11 \\ \hline
\multicolumn{1}{|c|}{\multirow{4}{*}{\textbf{CRDT \textit{Client-Server}}}} & CPU & 0.93 & 1.29 & 1.68 \\ \cline{2-5}
\multicolumn{1}{|c|}{} & RAM & 5.01 & 6.83 & 9.54 \\ \cline{2-5}
\multicolumn{1}{|c|}{} & Net In & 73.26 & 176.64 & 324.19 \\ \cline{2-5}
\multicolumn{1}{|c|}{} & Net Out & -71.79 & -193.90 & -391.66 \\ \hline
\multicolumn{1}{|c|}{\multirow{4}{*}{\textbf{OT \textit{Client-Server}}}} & CPU & 4.85 & 14.18 & 24.33 \\ \cline{2-5}
\multicolumn{1}{|c|}{} & RAM & 30.92 & 41.88 & 69.99 \\ \cline{2-5}
\multicolumn{1}{|c|}{} & Net In & 52318.08 & 167768.04 & 308540.87 \\ \cline{2-5}
\multicolumn{1}{|c|}{} & Net Out & -26494.76 & -84932.31 & -156178.28 \\ \hline
\end{tabular}
\end{table}

Aspek \textit{lightweight} lainnya yang perlu dipertimbangkan ialah penggunaan \textit{resource} atau sumber daya pada server yang dideskripsikan pada tabel~\ref{tab:resource-server-1}. Dalam eksperimen ini, server \textit{signalling} WebRTC pada CRDT \textit{peer-to-peer} menggunakan \textit{resource} RAM dan CPU yang minim. Berbeda halnya dengan arsitektur \textit{client-server}, selain berperan dalam memelihara koneksi kelompok klien dalam suatu ruangan atau kelompok jaringan, server juga menjaga konsistensi replika pada setiap klien. Data yang dikirim dan diterima pada sebuah \textit{peer} pada arsitektur \textit{peer-to-peer} memiliki ukuran 1.5 hingga 2.5 kali lebih besar dibandingkan \textit{bandwidth} data server pada CRDT \textit{client-server}.

\begin{figure}
 \centering
 \includegraphics[width=4.08cm]{./assets/skripsi/crdt1}
 \includegraphics[width=4.8cm]{./assets/skripsi/crdt2}
 \includegraphics[width=4.8cm]{./assets/skripsi/crdt3}
 \caption{Ilustrasi Sinkronisasi PeerToCP Variasi CRDT Berbasis \textit{Peer-To-Peer}}
 \label{fig:syncCRDT}
\end{figure}

Implementasi CRDT pada library Yjs menggunakan \textit{logical timestamp} berupa \textit{state vector} yang mencatat versi dari setiap pengguna dalam sebuah jaringan. Saat dua buah pengguna saling bersinkronisasi, hanya perbedaan antara dua buah versi yang didapat dari \textit{state vector} tersebut yang akan dikirimkan. Dalam beberapa kasus, bila perubahan dokumen terlalu berbeda, seluruh konten dokumen secara langsung dapat dikirimkan untuk menghemat ukuran. Perubahan dokumen dari \textit{state vector} ini bersifat komutatif dan idempoten, yang berarti perubahan yang sama dapat diterapkan berulang kali dan dengan urutan sembarang tanpa mengubah sifat konvergensi dokumen. Karena adanya sifat dari CRDT ini, pengiriman data kepada setiap klien pada CRDT tidak dipengaruhi banyaknya operasi yang dilakukan, melainkan cenderung dipengauruhi ukuran perubahan dokumen yang terjadi saat sinkronisasi dilakukan.

Pada gambar ilustrasi~\ref{fig:syncCRDT}, apabila klien A melakukan sinkronisasi dengan klien B terlebih dahulu, kemudian diikuti dengan sinkronisasi klien A dengan klien C dan sinkronisasi klien B dengan klien C, maka klien C hanya perlu memberikan perbedaan yang belum didapatnya dari klien A. Secara khusus, apabila terdapat operasi penghapusan pada objek di struktur data yang belum disinkronisasi, konten objek tersebut tidak perlu dipropagasi. Kasus ini merupakan salah satu optimisasi Yjs yang dapat mengurangi total \textit{bandwidth} yang perlu ditransmisikan antar pengguna pada variasi CRDT.

Dalam beberapa referensi implementasi, penyimpanan dokumen pada arsitektur \textit{client-server} dapat ditambahkan juga dengan basis data eksternal untuk dapat menampung lebih banyak dokumen pada suatu kelompok jaringan. Dalam eksperimen ini, basis data pada setiap variasi tidak menggunakan teknologi pihak ketiga dan hanya menggunakan struktur data bawaan sederhana pada server. Sementara pada arsitektur \textit{client-server}, persistensi dokumen dapat dilakukan dengan menyimpan dokumen pada penyimpanan lokal pada setiap \textit{peers}.


\begin{figure}
 \centering
 \includegraphics[width=15cm]{./assets/skripsi/Compare}
 \caption{Ilustrasi Perbandingan Arsitektur \textit{Peer-To-Peer} dan \textit{Client-Server} Secara Berurutan}
 \label{fig:compare}
\end{figure}

\textit{Drawback} dari transmisi rendah pada aplikasi pengguna CRDT \textit{client-server} ialah transmisi yang lebih tinggi dari sisi server. Gambar~\ref{fig:compare} menunjukkan bahwa banyaknya koneksi kanal data pada \textit{peer-to-peer} tumbuh secara kuadratik. Total transmisi yang harus ditanggung oleh sebuah \textit{peer} pada arsitektur ini secara matematis akan bertumbuh lebih cepat terhadap jumlah pengguna pada suatu kelompok jaringan. Selain itu, bila dilihat dari algoritma resolusinya, variasi OT memiliki muatan jaringan paling besar dibandingkan dengan variasi lain yang memengaruhi utilisasi CPU dan RAM-nya. \textit{Resource} ini dipengaruhi oleh permintaan data masuk dan keluar yang cenderung lebih banyak. Dari aspek \textit{lightweight}, arsitektur \textit{client-server} pada OT tidak dipreferensi penggunaannya dan membutuhkan tinjauan kembali untuk dioptimisasi protokol pengiriman datanya. Selain itu, pemanfaatan variasi algoritma OT yang lebih kompleks dan optimal untuk dapat diteliti lebih lanjut untuk dapat menangani skenario perubahan intensif seperti skenario pertama.
