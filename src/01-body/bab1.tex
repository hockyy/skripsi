%-----------------------------------------------------------------------------%
\chapter{\babSatu}
\label{bab:1}
%-----------------------------------------------------------------------------%
Bab ini membahas latar belakang perkembangan teknologi komunikasi waktu nyata, aplikasinya di internet pada saat ini, serta tantangan dan hambatan perkembangan teknologi ini untuk dapat digunakan secara komersial bagi publik. Melalui latar belakang, disusun gagasan pengembangan dan implementasi sederhana sistem aplikasi yang memanfaatkan teknologi tersebut. Ada pula tujuan, manfaat, serta batasan penulisan untuk memberikan konteks visi, misi, dan lingkup pengembangan yang turut disampaikan pada bab ini.

%-----------------------------------------------------------------------------%


\section{Latar Belakang}
\label{sec:latarBelakang}
%-----------------------------------------------------------------------------%
Penggunaan aplikasi yang menghubungkan manusia, baik secara lisan maupun tulisan secara waktu nyata (\textit{real-time}) semakin marak digunakan. Pada situasi pandemi COVID-19 semasa penelitian ini, perusahaan dan institusi pendidikan mengadakan aktivitas jarak jauh dan membutuhkan suatu media komunikasi yang dapat diandalkan. Beberapa aplikasi yang umum digunakan ialah produk-produk Google Workspace, seperti Google Docs, Google Slides, Google Meet, ataupun dari pengembang lain, yakni WhatsApp, LINE, JetBrains Code With Me, Zoom, dan masih banyak lagi. Tidak lepas pula dari permainan video multipemain yang membolehkan pemainnya berinteraksi secara langsung dengan latensi yang cenderung rendah dan terasa seperti waktu nyata.

Teknologi komunikasi waktu nyata atau RTC (\textit{real-time communications}) merupakan sebuah istilah metode telekomunikasi untuk beberapa pengguna yang berinteraksi dengan latensi atau waktu jeda yang relatif rendah terhadap respons pengguna~\citep{arefin2013modified}. Teknologi RTC mulai menjadi fokus penelitian dan penggunaan sejak dikenalkannya teknologi WebSocket pada tahun 2008~\citep{fette2011websocket, reynolds2008web}. Penggunaan WebSocket dimanfaatkan untuk menurunkan latensi dan membuat komunikasi dalam waktu nyata dapat terwujud melalui implementasi yang optimal.

RTC (\textit{real-time communications}) dengan WebSocket diaplikasikan pada salah satu aplikasi penyuntingan dokumen yang umum digunakan yakni Google Docs yang fiturnya dikembangkan pada tahun 2010~\citep{googledocs1}. Google Docs menggunakan metode khusus yaitu OT (\textit{operational transformation}) yang mendukung berbagai kapabilitas kolaborasi~\citep{googledocs2,googledocs3}. Teknologi ini mulai berkembang dan diteliti pada tahun 1989~\citep{Ellis1989}. Sepanjang perkembangannya, ada beberapa isu kesalahan pada metode OT yang terdeteksi dan diselesaikan secara bertahap. Implementasi dari metode ini juga memiliki banyak variasi serta keuntungan dan kerugiannya masing-masing, baik dari aspek memori maupun waktu. Salah satu pengembang aplikasi Google Wave, yang merupakan teknologi pendahulu Google Docs membutuhkan waktu sekitar dua tahun untuk menyelesaikan implementasi dari metode OT ini~\citep{shareJS}. Meskipun membutuhkan waktu lama, Google Docs menjadi salah satu produk editor teks andalan dengan kolaborasi waktu nyata yang memanfaatkan arsitektur \textit{client-server} dan masih digunakan hingga saat ini.

Seiring perkembangan teknologi, pada tahun 2011 WebRTC dikenalkan sebagai protokol dan antarmuka pemrograman aplikasi yang mendukung komunikasi waktu nyata dua arah yang bekerja secara \textit{peer-to-peer}~\citep{dutton2012getting}. WebRTC menyediakan suatu protokol untuk membolehkan suatu pengguna untuk berkomunikasi langsung dengan setiap pengguna lain tanpa melalui server setelah melakukan proses \textit{signalling}. \textit{Signalling} merupakan istilah untuk melakukan inisiasi koneksi kanal data WebRTC melalui sebuah server~\citep{sredojev2015webrtc}. WebRTC dikembangkan dengan tujuan utama untuk melakukan komunikasi waktu nyata dengan data yang lebih besar, seperti media suara dan video~\citep{dutton2012getting}. Muatan ke server juga menjadi lebih ringan karena hanya digunakan untuk \textit{signalling}. Hal ini cenderung meningkatkan skalabilitas dan menyediakan lebih banyak ketersediaan jaringan WebRTC terhadap pengguna bila dibandingkan dengan \textit{client-server}.

WebRTC juga mulai diteliti untuk dapat digunakan dalam berbagai kasus penggunaan, salah satunya untuk penyuntingan dokumen secara berkolaborasi dan dalam waktu nyata. Metode OT yang umumnya digunakan pada arsitektur \textit{client-server} memiliki beberapa properti khusus pada sifat konvergensi hasil akhirnya yang menyebabkan implementasinya pada arsitektur \textit{peer-to-peer} akan lebih sulit~\citep{Sun2017}. Sesuai namanya, OT (\textit{operational transformation}) meresolusi dengan melakukan transformasi terhadap operasi-operasi penyuntingan yang dilakukan terhadap suatu dokumen~\citep{OTOverview1}. Pada arsitektur \textit{client-server}, metode ini mengandalkan suatu server sebagai satu sumber kebenaran data (\textit{single source of truth}) yang akan menyelesaikan resolusi setiap operasi yang masuk dari setiap kliennya.

Perkembangan struktur data yang disebut dengan \textit{conflict-free replicated data types} (CRDT) mulai menjadi alternatif untuk metode resolusi pada arsitektur \textit{peer-to-peer}. CRDT dikenalkan pada tahun 2006 dan mulai secara formal didefinisikan pada tahun 2011~\citep{Shapiro2011}. Struktur data ini digunakan pada komputasi sistem terdistribusi dan tidak membutuhkan koneksi yang selalu tersedia setiap saatnya bagi semua pengguna. Pada CRDT, resolusi dilakukan pada \textit{state} atau kondisi dokumen saat ini dan tidak melalui transformasi dari operasi-operasi penyuntingannya~\citep{CRDToverview1}. CRDT dapat pula digunakan pada arsitektur \textit{client-server}, dengan setiap resolusi diselesaikan pada server, dan perubahan akan di-\textit{broadcast} atau disebarkan pada setiap klien~\citep{Sun2019First}.

Kedua arsitektur, yakni \textit{client-server} dan \textit{peer-to-peer} memiliki banyak keuntungan dan kerugian. Reliabilitas dan sumber daya yang terfokus pada server dapat menjadi terbatas, namun lebih stabil karena performanya yang dipersiapkan oleh pengembang. Di lain sisi, arsitektur \textit{peer-to-peer} dipertimbangkan karena mengurangi waktu \textit{overhead} dalam berkomunikasi antarpenggunanya karena berhubungan langsung dan tidak melalui server. Jaringan \textit{peer-to-peer} dapat bekerja secara optimal saat banyaknya pengguna dalam suatu jaringan kecil~\citep{leibnitz2007peer, maly2003comparison}. Namun sebaliknya, jumlah koneksi dalam jaringan akan meningkat dengan kompleksitas $O(N^2)$ dengan susunan \textit{mesh} bila setiap pengguna terhubung dengan pengguna lainnya. Hal ini menyebabkan komunikasi data yang dilakukan akan meningkat dalam waktu kuadratik dan tidak efisien karena transmisi untuk data yang sama dilakukan untuk semua pengguna. Terlepas dari kelebihan dan kekurangan yang disampaikan, beberapa aplikasi editor kode kolaboratif waktu nyata yang ada saat ini dikembangkan dengan arsitektur tertentu sesuai dengan kebutuhannya. Misalnya terdapat editor kode Atom dengan \textit{plugin} Teletype, Brackets, dan JetBrains Code With Me yang berbasis \textit{peer-to-peer}, atau pun codecollab, codeshare, dan Google Docs yang berbasis \textit{client-server}.

Editor kode kolaboratif lebih lanjut dapat dikembangkan menjadi suatu \textit{environment} yang membolehkan kolaborasi untuk digunakan dalam pemrograman kompetitif. Pemrograman kompetitif merupakan cabang olahraga pemrograman yang diperlombakan secara individu atau berkelompok untuk mengerjakan soal komputasional dengan batasan waktu dan memori tertentu. Pada pemrograman kompetitif, kode yang diedit umumnya bersifat sebuah berkas tunggal (\textit{single file}), yang dapat dikompilasi atau dijalankan tersendiri. Beberapa bahasa yang umum digunakan antara lain C, C++, Python, dan Java. Penelitian ini akan membahas pengembangan sebuah aplikasi editor kode, akses kompilator, dan \textit{shell} kolaboratif yang mendukung fitur pemrograman bersama dalam waktu nyata untuk setiap pengguna dalam sebuah jaringan. Penelitian ini juga menunjukkan hasil analisis \textit{benchmarking} performa PeerToCP yang menggunakan metode resolusi \textit{operational transformation} berbasis \textit{client-server}, struktur data CRDT berbasis arsitektur \textit{peer-to-peer}, serta struktur data CRDT pula, namun berbasis \textit{client-server}.

%-----------------------------------------------------------------------------%


\section{Pertanyaan Penelitian}
\label{sec:definisiMasalah}
%-----------------------------------------------------------------------------%
Berdasarkan paparan latar belakang dan rumusan masalah yang disampaikan, berikut adalah pertanyaan-pertanyaan yang hendak dijawab dari penelitian.

\begin{enumerate}[noitemsep]
    \item Bagaimana implementasi dari beberapa variasi aplikasi PeerToCP yang menyediakan \textit{real-time collaborative code editor} dan \textit{shared shell}?
    \item Bagaimana perbandingan performa, latensi, dan kebutuhan \textit{resource} sistem pengguna dan server PeerToCP berbasis WebRTC dengan variasinya yang lain?
    \item Berdasarkan hasil analisis dan evaluasi, apa keuntungan PeerToCP dengan arsitektur \textit{peer-to-peer} dan bagaimana pertimbangan penggunaan sistem dengan variasi tertentu?
\end{enumerate}

%-----------------------------------------------------------------------------%

\section{Batasan Penelitian}
\label{sec:batasanMasalah}
%-----------------------------------------------------------------------------%
Pertanyaan penelitian yang disampaikan pada subbab sebelumnya dibatasi oleh beberapa batasan penelitian. Pembatasan ini untuk memberikan kejelasan cakupan dan jangkauan penelitian yang disampaikan dalam karya ini. Dalam penelitian ini, \textit{user interface} dan \textit{user experience} dari bagian tampilan aplikasi tidak akan diuji dengan metode interaksi manusia dan komputer. Evaluasi pada penelitian ini juga diujikan dengan asumsi koneksi dan jaringan internet yang tidak terkendala, serta jumlah pengguna yang terbatas seperti pemrograman kompetitif pada umumnya. PeerToCP masih menggunakan \textit{library}, modul, dan \textit{framework} yang sudah tersedia dengan modifikasi dan penyesuaian seperlunya dalam proses pengembangan sistem aplikasi. Untuk setiap variasi PeerToCP, terdapat beberapa perbedaan fitur dan perilaku minor (tidak signifikan terhadap operasi yang akan dievaluasi) yang diabaikan.

%-----------------------------------------------------------------------------%


\section{Tujuan Penelitian}
\label{sec:tujuan}
%-----------------------------------------------------------------------------%
Terlepas dari batasan dan cakupannya, penelitian ini bertujuan untuk mewujudkan potensi dari teknologi \textit{real-time communications} yakni WebSocket dan WebRTC dalam bentuk aplikasi PeerToCP, sebuah \textit{environment} pemrograman kompetitif kolaboratif \textit{real-time}. Bersama tujuan tersebut, detail implementasi dipaparkan secara detail untuk menerangkan hambatan dan solusi yang diambil dalam pengembangan aplikasi ini. Penelitian ini juga menunjukkan evaluasi perbandingan variasi implementasi dari PeerToCP dengan uji skenario berbagai operasi esensial yang dapat dilakukan.


\section{Manfaat Penelitian}
\label{sec:manfaat}
%-----------------------------------------------------------------------------%
Penelitian ini diharapkan dapat memberikan gambaran umum dan bentuk prototipe dasar beberapa teknologi \textit{real-time communication}, seperti teknologi WebRTC (\textit{Web Real-Time Communication}) dan beberapa metode resolusi sinkronisasi replika data dalam beberapa pengguna dalam sebuah jaringan. Lebih lanjut, aplikasi ini berpotensi untuk dikembangkan lebih lanjut ke tahap produksi dan digunakan secara komersial. Penelitian ini juga didesain untuk menjadi acuan dalam penelitian tolok ukur lain terkait dengan performa arsitektur yang digunakan terhadap beberapa operasi komunikasi data tertentu, terutama dalam bentuk kolaborasi penyuntingan teks dan sinkronisasi data waktu nyata. Berangkat dari tujuan penelitian yang disampaikan sebelumnya pula, beberapa permasalahan dan hambatan pengembangan yang dipaparkan dapat diteliti lebih lanjut untuk membuat alternatif solusi yang lebih optimal terhadap solusi yang disampaikan pada penelitian ini.


\section{Sistematika Penulisan}

Untuk memberikan konteks penelitian yang padu dan terurut, laporan penelitian yang disampaikan dalam karya ini dibagi menjadi enam bagian, antara lain sebagai berikut.

\begin{enumerate}[noitemsep]
    \item Bab I Pendahuluan, memberikan konteks dasar dan pendahuluan dari penelitian, termasuk latar belakang, rumusan masalah yang terdiri dari pertanyaan penelitian dan batasannya, tujuan dan manfaat penelitian, serta sistematika penulisan keseluruhan tulisan.
    \item Bab II Tinjauan Pustaka, menyampaikan dasar-dasar studi dari pustaka yang berhubungan dengan penelitian yang dilakukan. Bab ini juga memberikan pengertian terminologi, teori, dan konsep tertulis terkait.
    \item Bab III Metodologi Penelitian, menerangkan metodologi yang digunakan dalam penelitian ini termasuk tahapan penelitian, desain implementasi, skenario pengujian, dan metrik evaluasi.
    \item Bab IV Implementasi, membahas detail implementasi dari aplikasi PeerToCP dengan berbagai variasinya.
    \item Bab V Hasil dan Pembahasan, menjelaskan hasil evaluasi terhadap pengujian performa yang dilakukan.
    \item Bab VI Penutup, memberikan kesimpulan serta saran akhir untuk perkembangan penelitian selanjutnya.
\end{enumerate}