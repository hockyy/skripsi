%
% Halaman Abstrak
%
% @author  Andreas Febrian
% @version 2.1.2
% @edit by Ichlasul Affan
%

\chapter*{Abstrak}
\singlespacing

\vspace*{0.2cm}

% Untuk conditional statement pembimbing dua
\def\blank{}

\noindent \begin{tabular}{l l p{10cm}}
	Nama&: & \penulis \\
	Program Studi&: & \program \\
	Judul&: & \judul \\
	Pembimbing&: & \pembimbingSatu \\
	\ifx\blank\pembimbingDua
    \else
        \ &\ & \pembimbingDua \\
    \fi
    \ifx\blank\pembimbingTiga
    \else
    	\ &\ & \pembimbingTiga \\
    \fi
\end{tabular} \\

\vspace*{0.5cm}

\noindent Aplikasi kolaboratif dalam waktu nyata menjadi bagian dari kehidupan manusia modern saat ini. Teknologi ini digunakan terutama untuk berkomunikasi dan meningkatkan produktivitas dengan mengurangi hambatan ruang dan waktu. Salah satu sistem aplikasi yang marak digunakan adalah editor dokumen yang dapat digunakan beberapa pengguna secara bersamaan. Dilatar belakangi oleh motivasi tersebut, penelitian ini memaparkan sebuah aplikasi editor kode kolaboratif \textit{local-first} berbasis \textit{peer-to-peer} yang diimplementasi dengan WebRTC dan CRDT. Selain itu, aplikasi ini menyertai \textit{shell} bersama yang dapat dijalankan oleh salah satu pengguna dan digunakan oleh setiap pengguna lain dalam suatu kelompok jaringan. Terdapat beberapa variasi arsitektur \textit{backend} pada aplikasi yang dibandingkan dalam penelitian ini. Dari segi algoritma dalam menjaga konsistensi dokumen, dua pendekatan berbeda yang diteliti yakni algoritma OT (\textit{operational transformation}) dan metode yang memanfaatkan struktur data CRDT (\textit{conflict-free replicated data types}). Dari segi arsitektur jaringan, penelitian ini mengevaluasi CRDT berbasis \textit{client-server}, CRDT berbasis \textit{peer-to-peer}, serta OT berbasis\textit{client-server}. Keterbatasan OT yang diimplementasi pada penelitian ini membutuhkan suatu sumber kebenaran berupa server, sehingga OT berbasis \textit{peer-to-peer} tidak dievaluasi. Penelitian ini menemukan bahwa variasi implementasi CRDT \textit{peer-to-peer} yang diujikan memiliki performa lebih baik untuk sejumlah pengguna $n \leq 8$. Selain itu, \textit{signalling server} pada variasi ini menggunakan \textit{resource} yang minim, sehingga lebih optimal untuk kelompok jaringan yang lebih banyak. Terlepas dari hal tersebut, variasi CRDT \textit{client-server} memiliki pertumbuhan kompleksitas waktu yang lebih rendah terhadap jumlah pengguna dalam satu kelompok jaringan dan penggunaan \textit{resource} pada sistem klien yang lebih rendah, namun lebih tinggi pada servernya. Sehingga variasi CRDT \textit{client-server} dapat dipertimbangkan penggunaannya ketika terjadi masalah saat melakukan inisialiasi jaringan \textit{peer-to-peer} atau jumlah pengguna dalam suatu kelompok jaringan jauh lebih banyak dari eksperimen yang dilakukan pada penelitian ini.\\

\vspace*{0.2cm}

\noindent Kata kunci: \\ WebRTC, WebSocket, \textit{local-first}, waktu nyata, editor kode, \textit{conflict-free replicated data types}, \textit{operational transformation}, \textit{shell} bersama, Yjs, Electron \\

\setstretch{1.4}
\newpage
