%
% Halaman Abstrak
%
% @author  Andreas Febrian
% @version 2.1.2
% @edit by Ichlasul Affan
%

\chapter*{Abstrak}
\singlespacing

\vspace*{0.2cm}

% Untuk conditional statement pembimbing dua
\def\blank{}

\noindent \begin{tabular}{l l p{10cm}}
	Nama&: & \penulis \\
	Program Studi&: & \program \\
	Judul&: & \judul \\
	Pembimbing&: & \pembimbingSatu \\
	\ifx\blank\pembimbingDua
    \else
        \ &\ & \pembimbingDua \\
    \fi
    \ifx\blank\pembimbingTiga
    \else
    	\ &\ & \pembimbingTiga \\
    \fi
\end{tabular} \\

\vspace*{0.5cm}

\noindent Aplikasi kolaboratif dalam waktu nyata menjadi bagian dari kehidupan manusia modern saat ini. Teknologi ini terutama digunakan untuk berkomunikasi dan meningkatkan produktivitas tanpa batas ruang dan waktu. Salah satu sistem aplikasi yang marak digunakan adalah editor dokumen yang dapat digunakan beberapa pengguna secara bersamaan. Di latar belakangi oleh motivasi tersebut, penelitian ini memaparkan sebuah aplikasi editor kode kolaboratif \textit{local-first} yang disertai dengan \textit{shell} bersama yang dapat dijalankan oleh salah satu pengguna dan digunakan oleh setiap pengguna lain dalam suatu kelompok jaringan. Terdapat beberapa variasi arsitektur jaringan aplikasi yang menyusun kerangka \textit{backend} aplikasi ini. Dari segi algoritma dalam menjaga konsistensi dokumen, terdapat dua pendekatan berbeda yang dievaluasi dalam penelitian ini, yakni implementasi algoritma OT (\textit{operational transformation}) sederhana dan struktur data CRDT (\textit{conflict-free replicated data types}). Dari segi arsitektur jaringan, akan dievaluasi pula CRDT yang berbasis \textit{client-server} dan \textit{peer-to-peer}. Penelitian ini menemukan bahwa variasi implementasi CRDT \textit{peer-to-peer} yang diujikan memiliki performa lebih baik untuk sejumlah pengguna $n \leq 8$. Selain itu, \textit{signalling server} pada variasi ini menggunakan \textit{resource} yang minim, sehingga lebih optimal untuk kelompok jaringan yang lebih banyak. Terlepas dari hal tersebut, variasi CRDT \textit{client-server} memiliki pertumbuhan kompleksitas waktu yang lebih rendah terhadap jumlah pengguna dalam satu kelompok jaringan dan penggunaan \textit{resource} sistem klien yang lebih rendah, namun lebih tinggi di servernya. Sehingga diperkirakan untuk jumlah pengguna $n$ dalam suatu kelompok jaringan yang lebih banyak, variasi \textit{client-server} dapat lebih dipertimbangkan.  \\

\vspace*{0.2cm}

\noindent Kata kunci: \\ WebRTC, WebSocket, \textit{local-first}, waktu nyata, editor kode, \textit{conflict-free replicated data types}, \textit{operational transformation}, \textit{shell} bersama, Yjs, Electron \\

\setstretch{1.4}
\newpage
