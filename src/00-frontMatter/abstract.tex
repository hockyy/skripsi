%
% Halaman Abstract
%
% @author  Andreas Febrian
% @version 2.1.2
% @edit by Ichlasul Affan
%

\chapter*{ABSTRACT}
\singlespacing

\vspace*{0.2cm}

% Untuk conditional statement pembimbing dua
\def\blank{}

\noindent \begin{tabular}{l l p{11.0cm}}
	Name&: & \penulis \\
	Study Program&: & \studyProgram \\
	Title&: & \judulInggris \\
	Counsellor&: & \pembimbingSatu \\
	\ifx\blank\pembimbingDua
	\else
		\ &\ & \pembimbingDua \\
	\fi
	\ifx\blank\pembimbingTiga
	\else
		\ &\ & \pembimbingTiga \\
	\fi
\end{tabular} \\

\vspace*{0.5cm}

\noindent Real-time collaborative applications are becoming a part of modern human life today. These technologies are used primarily to communicate and increase productivity by reducing time and space barriers. One of the widely used application systems is a document editor that can be used by multiple users simultaneously. Based on this motivation, this research presents a peer-to-peer and local-first collaborative code editor application implemented with WebRTC and CRDT. In addition, the application includes a shared shell that can be run by one user and used by every other user in a network group. There are several variations of backend architecture in the applications compared in this study. In terms of algorithms for maintaining document consistency, two different approaches were evaluated, namely OT (operational transformation) algorithm and CRDT (conflict-free replicated data types) data structure. In terms of network architecture, this study assessed client-server based CRDT, peer-to-peer based CRDT, and client-server based OT. The limitation of OT implemented in this research is that it requires a single source of truth in the form of a server, so peer-to-peer-based OT was not evaluated. This study found that the peer-to-peer based CRDT variation tested performed better for a number of users $n \leq 8$. Moreover, the signaling server in this variation uses minimal resources, making it more optimal for larger network groups. However, the client-server CRDT variation has lower time complexity growth with respect to the number of users in a network group and lower resource usage on the client system, but higher on the server. Therefore, the client-server CRDT variation's usage can be considered when there are problems initializing a peer-to-peer network or the number of users in a network group is much larger than the experiments conducted in this study. \\

\vspace*{0.2cm}

\noindent Key words: \\ WebRTC, WebSocket, local-first, real-time, code editor, conflict-free replicated data types, operational transformation, shared shell, Yjs, Electron \\

\setstretch{1.4}
\newpage
