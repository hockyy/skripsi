%
% Halaman Abstract
%
% @author  Andreas Febrian
% @version 2.1.2
% @edit by Ichlasul Affan
%

\chapter*{ABSTRACT}
\singlespacing

\vspace*{0.2cm}

% Untuk conditional statement pembimbing dua
\def\blank{}

\noindent \begin{tabular}{l l p{11.0cm}}
	Name&: & \penulis \\
	Study Program&: & \studyProgram \\
	Title&: & \judulInggris \\
	Counsellor&: & \pembimbingSatu \\
	\ifx\blank\pembimbingDua
	\else
		\ &\ & \pembimbingDua \\
	\fi
	\ifx\blank\pembimbingTiga
	\else
		\ &\ & \pembimbingTiga \\
	\fi
\end{tabular} \\

\vspace*{0.5cm}

\noindent Real-time collaborative applications are becoming a part of modern human life today. This technology is mainly used to communicate and increase productivity removing time and space limits. One of the widely used application is a document editor that can be used by multiple users simultaneously. Motivated by this, this research presents a local-first collaborative code editor application with a shared shell that can be run by one user and used by every other user in a network group. There are several variations of application network architecture that make up the backend framework of this application. In terms of algorithms for maintaining document consistency, two different approaches are evaluated in this research, namely the implementation of a simple OT (operational transformation) algorithm and a CRDT (conflict-free replicated data types) data structure. In terms of network architecture, client-server and peer-to-peer variation of CRDT are also evaluated. This research found that the peer-to-peer CRDT variation performed better for $n \leq 8$ number of users. In addition, the signalling server in this variation uses minimal resources, making it more optimal for larger number of network groups. Apart from that, the client-server CRDT variation has a lower growth in time complexity with respect to the number of users in a network group. This variation also uses lower resources on the client system, despite higher needs on the server. So the client-server variation can be considered, for a larger number of users $n$ in a network group. \\

\vspace*{0.2cm}

\noindent Key words: \\ WebRTC, WebSocket, local-first, real-time, code editor, conflict-free replicated data types, operational transformation, shared shell, Yjs, Electron \\

\setstretch{1.4}
\newpage
