%-----------------------------------------------------------------------------%
\chapter*{\kataPengantar}
%-----------------------------------------------------------------------------%
\pagestyle{first-pages}

Puji syukur kita panjatkan ke hadirat Tuhan Yang Maha Esa, karena rahmat dan anugerah-Nya, penulis dapat menyelesaikan skripsi yang berjudul “\judul” yang menjadi salah satu syarat kelulusan dalam menempuh pendidikan Sarjana Ilmu Komputer di  Fakultas Ilmu Komputer, Universitas Indonesia. Penulis juga ingin berterima kasih kepada pihak-pihak lain, khususnya kepada:

\begin{itemize}
    \setlength\itemsep{-0.5em}
    \item kedua orang tua, keluarga, dan kakak-kakak penulis yang mendukung proses perkuliahan sembari menyelesaikan skripsi ini;
    \item Bapak Muhammad Hafizhuddin Hilman S.Kom., M.Kom., Ph.D. selaku dosen pembimbing tugas akhir yang senantiasa melakukan supervisi kepada penulis dan memberikan pengetahuan setiap pekannya;
    \item Ibu Dr. Putu Wuri Handayani, S.Kom., M.Sc., Ibu Dipta Tanaya, S.Kom., M.Kom., dan Ibu Dr. Eng. Laksmita Rahadianti S.Kom., M.Sc. selaku dosen yang mengarahkan penulis dalam ilmu metodologi penelitian dan penulisan ilmiah;
    \item seluruh dosen yang dengan sabar mengajarkan ilmunya selama menempuh studi di Fakultas Ilmu Komputer, Universitas Indonesia;
    \item serta teman-teman penulis: Eko, Joni, Samuel, Kak Prabowo, Pikatan, Kenta, Andre, Irfancen, Raihan, Rafi, Aimar, Lucky, Novaryo, Kak Rey, dan teman-teman lain yang tidak dapat penulis sebutkan satu per satu namanya, karena setia menemani dan memberikan dukungan mental kepada penulis.
\end{itemize}

Penulis juga menyadari bahwa masih terdapat kesalahan dan kekurangan dalam penulisan karya ilmiah ini. Penulis berharap karya tulis ini dapat memberikan manfaat dan inspirasi untuk pengembangan dan peradaban ilmu pengetahuan teknologi dan informatika dunia, terutama bangsa Indonesia.

\vspace*{0.1cm}
\begin{flushright}
Depok, \tanggalSiapSidang\\[0.1cm]
\vspace*{1.5cm}
\penulis

\end{flushright}
