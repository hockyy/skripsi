%-----------------------------------------------------------------------------%
\addappendix{Kode dan Implementasi Aplikasi}
\chapter*{Lampiran 1: Kode dan Implementasi Aplikasi}
\label{appendix:implementation}
%-----------------------------------------------------------------------------%
Setiap kode, implementasi aplikasi pengguna, serta eksperimen pengujian pada penelitian ini dapat diakses pada tautan repositori Github sebagai berikut \texttt{\url{https://github.com/hockyy/peertocp}}. Setiap variasi dari PeerToCP dipisahkan berdasarkan \textit{branch}: \texttt{crdt-cs} yang merupakan variasi CRDT dengan arsitektur \textit{client-server}, \texttt{ot-cs} yang merupakan variasi \textit{operational transformation} dengan arsitektur \textit{client-server}, serta \texttt{crdt-p2p} yang merupakan variasi CRDT dengan arsitektur \textit{peer-to-peer}. Panduan untuk menjalankan kembali pengujian dan skenarionya terdapat pada bagian README.md dari branch \texttt{crdt-p2p} yang ditampilkan sebagai \textit{branch} utama dari \textit{repository}. Implementasi dari server dan modifikasi \textit{provider} koneksi yang digunakan pada branch masing-masing dapat diakses pada tautan repositori Github:

\begin{itemize}[nolistsep, noitemsep]
    \item \texttt{crdt-cs}, dapat diakses pada \texttt{\url{https://github.com/hockyy/y-websocket}};
    \item \texttt{crdt-p2p}, dapat diakses pada \texttt{\url{https://github.com/hockyy/y-webrtc}};
    \item \texttt{ot-cs}, dapat diakses pada \texttt{\url{https://github.com/hockyy/peertocp-ot-server}}.
\end{itemize}

Dalam melakukan eksperimen, setiap \textit{instance} diinisialisasi dengan beberapa aplikasi dan pengaturan melalui perintah-perintah sebagai berikut.

\begin{minted}[tabsize=2,breaklines]{bash}
sudo apt install -y git wget screen nginx python-is-python3 g++ make
sudo apt install -y build-essential clang libdbus-1-dev libgtk2.0-dev \
                       libnotify-dev libgnome-keyring-dev libgconf2-dev \
                       libasound2-dev libcap-dev libcups2-dev libxtst-dev \
                       libxss1 libnss3-dev gcc-multilib g++-multilib libasound2 xvfb \
export DISPLAY=192.168.0.5:0.0
curl https://my-netdata.io/kickstart.sh > /tmp/netdata-kickstart.sh && sh /tmp/netdata-kickstart.sh
curl -o- https://raw.githubusercontent.com/nvm-sh/nvm/v0.39.2/install.sh | bash
source ~/.bashrc
nvm install 16
nvm use 16
git clone [URL]

# Menggunakan xvfb karena debian tidak ada desktop
xvfb-run npm start
\end{minted}

Data diambil dan diproses pada perangkat lokal dengan \textit{script} pengunduhan log hasil evaluasi sebagai berikut.

\begin{minted}[tabsize=2,breaklines]{bash}
netd () {
    curl "http://$1:19999/api/v1/data?chart=apps.mem&dimension=node \
          &after=$2&points=0&group=average&gtime=0 \
          &timeout=0&format=csv&options=seconds" \
          > mem-$3.csv
    curl "http://$1:19999/api/v1/data?chart=system.ip \
          &after=$2&points=0&group=average&gtime=0 \
          &timeout=0&format=csv&options=seconds" \
          > network-$3.csv
    curl "http://$1:19999/api/v1/data?chart=apps.cpu&dimension=node \
          &after=$2&points=0&group=average&gtime=0 \
          &timeout=0&format=csv&options=seconds" \
          > cpu-$3.csv
}

scpd () {
    scp -r hocky@$1:~/peertocpnext/out/ ./$2
}
\end{minted}

Setiap server menggunakan NGINX dengan konfigurasi sebagai berikut.

\begin{minted}[tabsize=2,breaklines]{nginx}
location / {
    # First attempt to serve request as file, then
    # as directory, then fall back to displaying a 404.
    # try_files $uri $uri/ =404;
    proxy_pass http://127.0.0.1:3000;

    proxy_http_version  1.1;
    proxy_set_header Upgrade $http_upgrade;
    proxy_set_header Connection "upgrade";
    proxy_set_header Host $http_host;
    proxy_set_header X-Real-IP $remote_addr;
}
\end{minted}

Hasil visualisasi dan interpretasi dari eksperimen yang disampaikan pada penelitian ini dapat diakses mellaui tautan repositori Github sebagai berikut \texttt{\url{https://github.com/hockyy/peertocp-benchmark}}.